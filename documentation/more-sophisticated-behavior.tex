\section{More-sophisticated behavior}


\subsection{Selfstudy-Questions OOP6}

\subsubsection{Chapter 5.2 - The TechSupport system}

\subsubsection*{Exercise 1}
\textit{Solve the exercise 5.1}

\subsubsection*{Exercise 2}
\textit{At page 157 and 158 there is a Method start() that uses a 
while-loop. Create a code snippet with the same functionality by using a
do-while loop.}

\subsubsection{Chapter 5.3 - Reading class documentation}

\subsubsection*{Exercise 3}
\textit{Solve the exercises 5.2 to 5.5 as well as 5.7 to 5.11}

\subsubsection*{Exercise 4}
\textit{In which package do you suppose the class FileWriter?
Check your guess with the Java API documentation.}

\subsubsection*{Exercise 5}
\textit{With the class BufferReader you can read files line by line.
How does that work? You can find the answer in the Java API documentation.}

\subsubsection{Chapter 5.4 - Adding random behavior}

\subsubsection*{Exercise 6}
\textit{Solve the exercises 5.12 and 5.13}

\subsubsection*{Exercise 7}
\textit{Solve the exercise 5.15}

\subsubsection*{Exercise 8}
\textit{Solve the exercise 5.18}

\subsubsection{Chapter 5.5 - Packages and import}

\subsubsection*{Exercise 9}
\textit{Solve the exercises 5.21 and 5.22}



\subsection{Selfstudy-Questions ALG1}

\subsubsection*{Exercise 10}
\textit{Describe in words or as pseudocode one algorithm per problem.}
\begin{enumerate}[label={(\alph*)}]
	\item Calculate the product of two integers without a multiplication-operator.
	\item Find the lowest number out of a sequence like $S = {4,-1,50,10,0,1,-2,5,10}$
\end{enumerate}

\subsubsection*{Exercise 11}
\textit{Implement an algorithm to calculate the greatest common divisor
with the modulo operator (see ALG1 presentation, page 12).}
\begin{enumerate}[label={(\alph*)}]
	\item Test your algorithm with some examples.
	\item Change your algorithm so that you don't use the modulo operator. 
		Change the modulo operation with a combined expression and test 
		your implementation again. \\
		\textit{Hint: The modulo operator can be substuituted by a 
			sequence of subtractions that is finished as the 
			result is smaller than the subtrahend.}
\end{enumerate}

\subsubsection*{Exercise 12}
\textit{What's the order of the algorithm to calculate the n-th pseudo random 
number z (see OOP6 presentation page 28 and ALG1 presentation page 15)?}
\[ z_{n+1} = (a \cdot Z_n + r) \% m \]
