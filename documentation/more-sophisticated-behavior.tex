\section{More-sophisticated behavior}

\subsection{Selfstudy-Questions OOP6}

\subsubsection{Chapter 5.2 - The TechSupport system}

\subsubsection*{Exercise 1}
\textit{Solve the exercise 5.1}\\

\lstinputlisting{../workspace/TechSupport/src/support/Main.java}

\subsubsection*{Exercise 2}
\textit{At page 157 and 158 there is a Method start() that uses a 
while-loop. Create a code snippet with the same functionality by using a
do-while loop.}\\

\begin{lstlisting}
public void start()
{
	boolean finished = false;

	printWelcome();

	do{
		String input = reader.getInput();
		if(input.startsWith("bye")){
			finished = true;
		} else{
			String response = responder.generateResponse();
			System.out.println(response);
		}
	}while(!finished)
	
	printGoodbye();
}
\end{lstlisting}

\subsubsection{Chapter 5.3 - Reading class documentation}

\subsubsection*{Exercise 3}
\textit{Solve the exercises 5.2 to 5.5 as well as 5.7 to 5.11}\\

\textbf{5.2} The Java documentation has a clear design. Every class like 
String is described in different sections, to get as fast as possible to the
information needed. This documentation is arranged by the following 
sections:

\begin{itemize}
	\item Summary
	\begin{itemize}
		\item Nested
		\item Field
		\item Constructor
		\item Method
	\end{itemize}
	\item Detail
	\begin{itemize}
		\item Field
		\item Constructor
		\item Method
	\end{itemize}
\end{itemize}

\textbf{5.3} The class String has two methods called startsWith, where
one method has only one parameter (String) and the other has two parameters
(String and int).
The method that has only one parameter is returning \lstinline{true} if the
specified string is found at the very beginning of the specified String
object.
The other method with two parameters is returning \lstinline{true} if the
specified string is found after the specified offset. The offest is a
int value representing the number of characters.

Here is an example for the behavior. Both of these \lstinline{if} statements
are returning \lstinline{true}.

\begin{lstlisting}
String s = "Hi, my name is Avaj Elcaro.";

// check if the very beginning of the String object is "Hi"
if(s.startsWith("Hi")){
	System.out.println("found match at the very beginning");
}

// check if the String object has the string "my" after the 4th character
if(s.startsWith("my", 4)){
	System.out.println("found match after the 4th character");
\end{lstlisting}

\textbf{5.4} There is a method in the class String that checks if a 
String ends with a specified suffix.

\begin{lstlisting}
public boolean endsWith(String suffix)
\end{lstlisting}

\textbf{5.5} There is a method in the class String that returns the number
of unicode characters in a specified string.

\begin{lstlisting}
public int length()
\end{lstlisting}

\subsubsection*{Exercise 4}
\textit{In which package do you suppose the class FileWriter?
Check your guess with the Java API documentation.}\\

The class FileWriter is a part of the package java.io where the io stands
for Input-Output.

\subsubsection*{Exercise 5}
\textit{With the class BufferReader you can read files line by line.
How does that work? You can find the answer in the Java API documentation.}\\

We can use the method readLine() for that job. Here is an example form
\url{http://www.roseindia.net/java/beginners/java-read-file-line-by-line.shtml}

\begin{lstlisting}
import java.io.*;
class FileRead 
{
 	public static void main(String args[])
  	{
  		try{
  			// Open the file
			FileInputStream fstream = new FileInputStream("textfile.txt");
  			// Get the object of DataInputStream
  			DataInputStream in = new DataInputStream(fstream);
  			BufferedReader br = new BufferedReader(new InputStreamReader(in));
  			String strLine;
  			//Read File Line By Line
  			while ((strLine = br.readLine()) != null)   {
  				// Print the content on the console
				System.out.println (strLine);
  			}
  			//Close the input stream
  			in.close();
		}catch (Exception e){//Catch exception if any
			System.err.println("Error: " + e.getMessage());
  		}
	}
}
\end{lstlisting}

\subsubsection{Chapter 5.4 - Adding random behavior}

\subsubsection*{Exercise 6}
\textit{Solve the exercises 5.12 and 5.13}\\

\textbf{5.12} The Random class is from the package java.util. It is used to
generate a stream of pseudorandom numbers. An instance can be constructed
with the following code.

\begin{lstlisting}
import java.util.Random;

public class Main
{
	public static void main(String[] args)
	{
		// create an instance of Random
		Random rand = new Random();

		// print out 10 random numbers between 0 and 20
		for(int i = 0; i < 10; i++)
		{
			System.out.println(rand.nextInt(21));
		}
	}
}
\end{lstlisting}

\textbf{5.13} See 5.12

\subsubsection*{Exercise 7}
\textit{Solve the exercise 5.15}\\

If we use \lstinline{rand.nextInt(100)} we will get any integer number 
between 0 and 99.

\subsubsection*{Exercise 8}
\textit{Solve the exercise 5.18}\\

Simple example of a random response program.

\lstinputlisting{../workspace/Snippets/src/randomResponse/RandomResponse.java}

\subsubsection{Chapter 5.5 - Packages and import}

\subsubsection*{Exercise 9}
\textit{Solve the exercises 5.21 and 5.22}\\

\textbf{5.21} See Exercise 8 - 5.18

\textbf{5.22} If we implement the response generation like in the example code 
above, it will work for any number of responses in the ArrayList. This is 
because the code is flexible for the size of the ArrayList.

\newpage
\subsection{Selfstudy-Questions ALG1}

\subsubsection*{Exercise 10}
\textit{Describe in words or as pseudocode one algorithm per problem.}\\

\begin{enumerate}[label={(\alph*)}]
	\item \textit{Calculate the product of two integers without a 
		multiplication-operator.}\\

		One possible approach to solve this problem could be to
		use simple addition and iteration. This is because a 
		multiplication, for example $3 \cdot 5 = 15$, can be
		substituted with $(3 + 3 + 3 + 3 + 3) = (5 + 5 + 5) = 15$.
		\lstinputlisting{../workspace/Snippets/src/multiply/Multiplier.java}
		\lstinputlisting{../workspace/Snippets/src/multiply/Main.java}

		An alternative approach is to think of the factors as binary
		numbers. In the binary system we multiply also by a addition.
		See the following example: Lets multiply $11 \cdot 14 = 154$.
		In binary this looks like following:
		\[ \begin{array}{l r | l}
			& \texttt{1011}_b & 11_d \\
			\cdot & \textcolor{red}{1}
				\textcolor{green}{1}
				\textcolor{cyan}{1}
				\textcolor{blue}{0}_b & 14_d \\
			\hline \hline 
			& \texttt{0000}_b & \textcolor{blue}{0} \cdot 1011_b \\
			+ & \texttt{1011~}_b & \textcolor{cyan}{1} \cdot 1011_b \text{ and leftshift by 1} \\
			+ & \texttt{1011~~}_b & \textcolor{green}{1} \cdot 1011_b \text{ and leftshift by 2}\\
			+ & \texttt{1011~~~}_b & \textcolor{red}{1} \cdot 1011_b \text{ and leftshift by 3} \\
			\hline \hline
			= & \texttt{10011010}_b & 154_d
		\end{array} \]
		If we wanted to implement that alorithm, we would have to 
		operate binary, but in fact that would be not really 
		different form the fist approach.
	\item \textit{Find the lowest number out of a sequence like 
		$S = {4,-1,50,10,0,1,-2,5,10}$}\\
		
		\lstinputlisting{../workspace/Snippets/src/search/Main.java}
\end{enumerate}

\subsubsection*{Exercise 11}
\textit{Implement an algorithm to calculate the greatest common divisor
with the modulo operator (see ALG1 presentation, page 12).}\\

\lstinputlisting{../workspace/Snippets/src/euklid/Main.java}

\begin{enumerate}[label={(\alph*)}]
	\item Test your algorithm with some examples.
	\item Change your algorithm so that you don't use the modulo operator. 
		Change the modulo operation with a combined expression and test 
		your implementation again. \\
		\textit{Hint: The modulo operator can be substuituted by a 
			sequence of subtractions that is finished as the 
			result is smaller than the subtrahend.}
		\lstinputlisting{../workspace/Snippets/src/euklid2/Main.java}

\end{enumerate}

\subsubsection*{Exercise 12}
\textit{What's the order of the algorithm to calculate the n-th pseudo random 
number z (see OOP6 presentation page 28 and ALG1 presentation page 15)?}\\

\[ z_{n+1} = (a \cdot Z_n + r) \% m \]

\subsection{Programming Exercise}
\textit{Write a class FormLetter that is able to produce serial letters.
The class has an ArrayList of type FormalAddress (see OOP3). With the 
method addAddress() a new object of type FormalAddress is generated and
added to the ArrayList. With the method void print(int actualYear, String
subject, String textBody) a new serial letter is generated for all 
elements in the ArrayList. The letter contains a greeting, a subject and
of cource the text itself. This shall be printed to the console.}

\lstinputlisting{../workspace/FormLetter/src/letter/Main.java}
\lstinputlisting{../workspace/FormLetter/src/letter/FormLetter.java}
\lstinputlisting{../workspace/FormLetter/src/letter/FormalAddress.java}

\subsection{Team-Exercise}
\textit{Create a class ListIteratorApplication with the attribute of type
ArrayList for Strings. Fill the ArrayList with the words of the sentence
"With the iterator it is possible to tavers Lists back and forth".
\begin{enumerate}
	\item Program a method iterateDown() that is iterating through 
	the ArrayList. Use a Iterator for this method.
	\item Program a method iterateUp() that is iterating reverse
		through the ArrayList. Use a for-loop and the get()
		method form ArrayList.
	\item Program a method iterateBothWays() that is iterating
		through the ArrayList and then backwards. Use the
		ListIterator for this method and take a look at the
		API documentation of it.
	\item Optional Ecerxice: Use the StringTokenizer to fill the 
		ArrayList. An example is given by the OOP5 presentation
		on page 10. Take also a look at the API documentation.
\end{enumerate}}

\lstinputlisting{../workspace/ListApp/src/list/Main.java}
\lstinputlisting{../workspace/ListApp/src/list/ListIteratorApplication.java}
