\section{Objects and classes}

\subsection{Summary exercises}

\subsubsection*{Exercise 1.31}
\textit{What are the types of the following values?}\\

\begin{tabular}{ l l }
	0	& short, char, byte, int, long 	\\
	"hello"	& String			\\
	101	& short, char, byte, int, long	\\
	-1	& int, char, byte, int, long	\\
	true	& boolean			\\
	"33"	& String			\\
	3.1415	& float, double			\\
\end{tabular}

\subsubsection*{Exercise 1.32}
\textit{What would you have to do to add a new filed, for example one called 
	name, to a circle object?}\\

\lstinline{private String name;}

\subsubsection*{Exercise 1.33}
\textit{Write the signature of a method named send that has one parameter of 
	type String, and does not return a value.}\\

\lstinline{public void send(String foo)}

\subsubsection*{Exercise 1.34}
\textit{Write a signature for a method named average that has two parameters,
	both of type int, and returns an int value.}\\

\lstinline{public int average(int foo, int bar)}

\subsubsection*{Exercise 1.35}
\textit{Look at the book you are reading right now. Is it an object or class?
	If it is a class, name some objects. If it is an object, name its 
	class.}\\

The book is definitely an object, because it's a specific thing and in no way 
generic. The class could have a name like \lstinline{SchoolBook}, 
\lstinline{CodingBook} or just \lstinline{Book}.

\subsubsection*{Exercise 1.36}
\textit{Can an object have several different classes? Discuss.}\\

No it can't.
