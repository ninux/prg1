\section{Grouping objects}


\subsection{Selfstudy-Questions OOP5}

\subsubsection{Chapter 4.1 to 4.3 - An organizer for music files}

\subsubsection*{Exercise 1}
\textit{Solve the exercises 4.1 to 4.3}\\

\textbf{4.1}

\lstinputlisting{../workspace/Music-Organiser-V1/src/music/MusicOrganizer.java}
\lstinputlisting{../workspace/Music-Organiser-V1/src/music/Main.java}

\textbf{4.2} We don't get an error. As wee look at the code of the method
we'll see, that the method is examinating if the index is "valid". If not
it does not perform a remove action.

\lstinputlisting[firstline=58, lastline=63]{../workspace/Music-Organiser-V1/src/music/MusicOrganizer.java}

\textbf{4.3} The list is shifted, so that the previously second track is the
first track after removing the first track.

\subsubsection*{Exercise 2}
\textit{What do you understand by "Java-Package"?}\\

A Java-Package is just a collection of classes. It is a namespace to organize
classes.

\subsubsection*{Exercise 3}
\textit{You want to use the library-class ArrayList. What expression makes it
possible to use that library-class in your source code?}\\

\noindent
If we want to use a library-class, we have to import it to our source with

\lstinline{import java.nameOfTheLibraryClass;} \\

\noindent
So to use the ArrayList we just have to write in our source 

\lstinline{import java.util.ArrayList;} 

\subsubsection{Chapter 4.4 to 4.7 - Numbering within collections}

\subsubsection*{Exercise 4}
\textit{Solve the exercises 4.4 to 4.7}\\

\subsubsection*{Exercise 4.4, page 100}  
\textit{Write a declaration of a private field named library that can hold an
ArrayList. The elements of the ArrayList are of type Book.}\\

\lstinline{private ArrayList<Book> library = new ArrayList<String>();}

\subsubsection*{Exercise 4.5, page 100}
\textit{Write a declaration of a local variable called cs101 that can hold an
ArrayList of Student.}\\

\lstinline{private ArrayList<Student> cs101 = new ArrayList<Student>();}

I'm not really sure about the "local" in the exercise. Do I have to specify
that this is private or not?

\subsubsection*{Exercise 4.6, page 101}
\textit{Write a declaration of a private field called tracks for sorting
a collection of MusicTrack objects.}\\

\lstinline{private ArrayList<MusicTrack> tracks = new ArrayList<MusicTrack>();}

\subsubsection*{Exercise 4.7, page 101}
\textit{Write assignments to the library, cs101 and track variables (which
you defined in the previous three exercises) to create the appropriate 
ArrayList objects. Write them once without using diamond and once with
diamond notation if you are using Java 7 compiler.}\\

\begin{lstlisting}
library.add("Objects First with Java");
cs101.add("Leonardo DaVinci");
track.add("Free Software Song");
\end{lstlisting}

I'm not really sure what is the question here \dots

\subsubsection*{Exercise 5}
\textit{Solve the exercises 4.8 to 4.11}\\

\subsubsection*{Exercise 4.8, page 102}
\textit{If a collection stores 10 objects, what value would be returned from
a call to its size method?}\\

It would return 9.

\subsubsection*{Exercise 4.9, page 102}
\textit{Write a method call using get to return the fifth object stored in a
collection called items.}\\

\lstinline{items.get(4)}

\subsubsection*{Exercise 4.10, page 102}
\textit{What is the index of the last item stored in a collection of 15
objects?}\\

This would be 14.

\subsubsection*{Exercise 4.11, page 102}
\textit{Write a method call to add the object held in the variable 
favoriteTrack to a collection called files.}\\

\lstinline{addFavorite(favoriteTrack, files)}

I'm not really sure about this question \dots

\subsubsection*{Exercise 6}
\textit{Solve the exercises 4.12 to 4.13}\\

\subsubsection*{Exercise 4.12, page 103}
\textit{Write a method call to remove the third object stored in a
collection called dates.}\\

\lstinline{dates.remove(2)}

\subsubsection*{Exercise 4.13, page 103}
\textit{Suppose that an object is stored at index 6 in a collection.
What will be its index immediately after the objects at index 0 and 9
are removed?}\\

After removing index 0, the whole collection is shifted by $-1$, 
so the index of the element, which was at index 5 in the beginning
would now ba at $6-1=5$. Removing an index after the queried one has
non effect of the indexing, so it would still be 5.

\subsubsection*{Exercise 7}
\textit{Explanin the following declaration:}\\
\lstinline?private ArrayList<Balloon> list = new ArrayList<>();?\\

\subsubsection*{Exercise 8}
\textit{What is the connection between abstraction an ArrayLists?}\\

\subsubsection*{Exercise 9}
\textit{What is the difference of the methods remove() and get() on
ArrayLists?}\\

\subsubsection{Chapter 4.8 to 4.12 - The Iterator type}

\subsubsection*{Exercise 10}
\textit{Solve the exercises 4.18 to 4.19}\\

\subsubsection*{Exercise 11}
\textit{Solve the exercise 4.22}\\

\subsubsection*{Exercise 12}
\textit{Explain as detailed as possible the source code on page 108.}\\

\subsubsection*{Exercise 13}
\textit{Is it possible, that the body of an while-loop is never executed?}\\

\subsubsection*{Exercise 14}
\textit{Show two alternative expressions for no++}\\

\subsubsection*{Exercise 15}
\textit{An ArrayList can be traversed by an foreach-loop. Do you know other
ways to do the same?}\\

\subsubsection*{Exercise 16}
\textit{Is hasNext() a method of ArrayList or Iterator? How do you have to 
understand/interpret the return-value of hasNext()?}\\

\subsubsection{Chapter 4.14 - Summary of the music-organizer project}

\subsubsection*{Exercise 17}
\textit{DO NOT READ THIS CHAPTER, JUST READ THE CONCEPT-BOX AT PAGE 130.}\\

\subsubsection*{Exercise 18}
\textit{A variable that is declared for a classtype (or so called 
reference-variable) can store the special value null. Explain the situation 
with a drwing/sektch. What does it look like, if it's storing an object?}\\

\subsubsection{Chapter 4.15 to 4.17 - Summary}

\subsubsection*{Exercise 19}
\textit{Solve the exercises 4.62 to 4.65}\\

\subsubsection*{Exercise 20}
\textit{Solve the exercises 4.66 to 4.68}\\

\subsubsection*{Exercise 21}
\textit{What are the pros and cons of Arrays?}\\

\subsubsection*{Exercise 22}
\textit{How do you get the length of an Array?}\\

\subsubsection*{Exercise 23}
\textit{Solve the exercises 4.69, 4.71, 4.73 and 4.74}\\




