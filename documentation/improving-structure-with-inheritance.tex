\section{Improving structure with inheritance}


\subsection{Selfstudy-Questions OOP10}

\subsubsection{Chapter 8.1}

\subsubsection*{Exercise 1}
\textit{Solve the exercise 8.1} \\

\subsubsection{Chapter 8.3}

\subsubsection*{Exercise 2}
\textit{Solve the exercise 8.3} \\

\subsubsection{Chapter 8.4}

\subsubsection*{Exercise 3}
\textit{Solve the exercises 8.4 and 8.7} \\

\subsubsection{Chapter 8.5}

\subsubsection*{Exercise 4}
\textit{Solve the exercise 8.8} \\

\subsubsection{Chapter 8.6}

\subsubsection*{Exercise 5}
\textit{Solve the exercise 8.9} \\

\subsubsection{Chapter 8.7}

\subsubsection*{Exercise 6}
\textit{Solve the exercises 8.12 and 8.14} \\

\subsubsection{Chapter 8.11}

\subsubsection*{Exercise 7}
\textit{Solve the exercise 8.18} \\

\subsection{Selfstudy-Questions ALG5}

\subsubsection*{Exercise 8}
\textit{Sort the following sequence with the classic Quicksort algorithm
(according to the implementation in the class Quicksort.java on ILIAS).
Write down the steps and results in between.} \\

\begin{table}[h!]
	\centering
	\begin{tabular}{l c  c  c  c  c  c  c  c  c }
		Step & $[r_0]$ &$[r_1]$ &$[r_2]$ &$[r_3]$ &$[r_4]$ 
			&$[r_5]$ &$[r_6]$ &$[r_7]$ &$[r_8]$ \\ 
		\hline
		Initial sequence & 8 & 2 & 4 & 9 & 5 & 3 & 1 & 7 & 6 \\
		\hline
	\end{tabular}
	\caption{Quicksort example}
\end{table}

\subsubsection*{Exercise 9}
\textit{Check your results form exercise 8 by implementing a console 
	output for the given algorithm.} \\

\subsubsection*{Exercise 10}
\textit{When would you prefer an other sorting algorithm over the 
	Quicksort algorithm? Define two different scenarios and 
	explain your answer.} \\

\subsubsection*{Exercise 11}
\textit{Sort the following sequence with the Mergesort algorithm 
	(according to the implementation in Java, see Page 36).
	Write down the steps and results in between.} \\

\begin{table}[h!]
	\centering
	\begin{tabular}{l c  c  c  c  c  c  c  c  c }
		Step & $[r_0]$ &$[r_1]$ &$[r_2]$ &$[r_3]$ &$[r_4]$ 
			&$[r_5]$ &$[r_6]$ &$[r_7]$ &$[r_8]$ \\ 
		\hline
		Initial sequence & 8 & 2 & 4 & 9 & 5 & 3 & 1 & 7 & 6 \\
		\hline
	\end{tabular}
	\caption{Mergesort example}
\end{table}
