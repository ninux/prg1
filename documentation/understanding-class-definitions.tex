\section{Understanding class definitions}

\subsection{Start with Eclipse}
In the first chapter we've worked with the BlueJ IDE but now I want to check
Java-Coding with a common and popular Java-IDE like Eclipse
To get the BlueJ-Projects work with Eclipse there are some things that have to
be done.

\begin{enumerate}
\item Create a new project in Eclipse.
\item Import the source (BlueJ example-code).
\item Add a package-name to the source.
\item Create a main (replaces all interaction which were invoked by hand).
\end{enumerate}

\lstinputlisting[caption=TicketMachine,firstline=12,lastline=20]{../workspace/naive-ticket-machine/src/foobar/TicketMachine.java}

\lstinputlisting[caption=Main (TicketMachine)]{../workspace/naive-ticket-machine/src/foobar/Main.java}

\newpage
\subsection{Chapter Exercises}
\subsubsection*{Exercise 2.21}
\textit{Suppose that the class Pet has a field called name that is of type
	String. Write an assignment statement in the body of the following 
	constructor so that the name field will be initialized with the value
	of the constructor's parameter.}\\

\begin{lstlisting}
public Pet(String petsName)
{
	name = petsName;
}
\end{lstlisting}

\subsubsection*{Exercise 2.22 (challenge)}
\textit{The following object creation will result in the constructor of the
	Date class being called. Can you write the constructor's header?\\
	\indent \lstinline{new Date("March", 23, 1861)}\\
	Try to give meaningful names to the paramenters.}\\

\begin{lstlisting}
public Date(String month, int day, int year)
{
	...
}
\end{lstlisting}

\newpage
\subsection{Selfstudy-Questions OOP2}
\subsubsection*{Exercise 4}
\textit{A class is build by three essential components. What are they?}\\

\begin{itemize}
	\item Instance variables (member variables, attributes)
	\item constructor
	\item methods
\end{itemize}

\subsubsection*{Exercise 5}
\textit{What is the order of the three components?}\\

The order doesn't matter technically but there is a common convention:
\begin{enumerate}
	\item instance variables
	\item constructor
	\item methods
\end{enumerate}

\subsubsection*{Exercise 6}
\textit{What's their purpose?}\\

\begin{description}
	\item [instance variables] are holding data of an object. All of this
		data together builds the object's state.
	\item [constructor] is a special method that initializes objects.
	\item [methods] are sequences which are defining the object's 
		behaviour and characteristics.
\end{description}

\subsubsection*{Exercise 8}
\textit{What is a variable?}\\

A variable (or field) is a data storage inside an object that can be used for
persistent data storage (limited by the lifetime of the object).

\subsubsection*{Exercise 9}
\textit{What are the synonyms to instance variables?}\\

\begin{itemize}
	\item member variable
	\item attribute
	\item filed
	\item variable
\end{itemize}

\subsubsection*{Exercise 10}
\textit{What do you think where the term instance variable comes from?}\\

An instance is a realisation of an class by an object. The expression variable
is well defined an known in computer science and if a variable explicitly
belongs to an object, so it's clear that this is a variable of an instance or
instance variable.

\subsubsection*{Exercise 11}
\textit{How can you put comments into a Java-Code?}\\

There are different ways to add comments in a Java source file without having
trouble with the compiler.

\begin{itemize} 
	\item Use the single line comment by double slash.
\begin{lstlisting}
// this method return the speed
private void getSpeed()
\end{lstlisting}
	\item Use the multiline comment by slash-dot
\begin{lstlisting}
/**
 * This is a method that will return the
 * actual speed of the monstetruck that
 * is driven by the crazy clown IT .
 */
private void getSpeed()
\end{lstlisting}
\end{itemize}

\subsubsection*{Exercise 12 (important)}
\textit{With which access-modification do you declare instance variables
	usually? Is it \lstinline{private} or \lstinline{public}? Do you
	have a reason for your answer?}\\

Usually we declare instance variables as private. The reason for this is a 
common pattern that is used to get or set these data form outside the objects
by so called accessor and mutator methods (getSpeed, setSpeed, changeSpeed).

\subsubsection*{Exercise 13}
\textit{Explain the relation between a constructor and the state of an 
	onject.}\\

	The constructor is creating (initializing) an object and has nothing
	to do with the state of the object once it's set up.

\subsubsection*{Exercise 14}
\textit{How do we name constructors?}\\

Constructors are usually named after the class their used for.

\subsubsection*{Exercise 15}
\textit{What's the lifetyme of instance variables, how long are they
	reachable/accessable?}\\

The lifetime of variables is coupled to the lifetime of their objects.
As long as the object is alive the variables are also alive.

\subsubsection*{Exercise 16}
\textit{Why sould you (if possible) initialise instance variables explicit?}\\

If we don't initialize variables explicit the compiler will use default values
for the initialization. By explicit initialisation we don't have any 
disadwantage and it serves well to document what is sctually happening.

\subsubsection*{Exercise 17}
\textit{What's the defualt value which is given to a \lstinline{int} variavle
	by its initialisation?}\\

The default value for an \lstinline{int} is zero.

\subsubsection*{Exercise 19}
\textit{What's the use of parameters?}\\

Parameters provide additional information to a method or object. This is
useful in many ways.

\subsubsection*{Exercise 20}
\textit{What's the difference between a formal and a actual parameter?}\\

A formal parameter is a parameter that is defined as parameter but has no 
actual value corresponding. A actual parameter is a parameter with a 
specific value.

\subsubsection*{Exercise 21}
\textit{Is the following statement correct; "formal parameters are special
	variables"?}\\

Parameters are temporary and restricted variables because their space is
allocated by a call to the method or object and as soon as a value is 
transmitted to it. Once that call has completed its task, the formal
parameter disappears and the values in it are lost.

\subsubsection*{Exercise 22}
\textit{What's about the accessability of formal parameters?}\\

The accessability of parameters are limited to the lifetime of the task
which is creating them (method). Also parameter are only reachable from
inside the box that they are used in (like a local variable).

\subsubsection*{Exercise 23}
\textit{In which way this differs from instance variables?}\\

Instance variables have a lifetime that is identical with the lifetime of
their objects. Also parameters are only reachable from inside the block,
instance variables are reachable from everywhere inside the class.

\subsubsection*{Exercise 24}
\textit{How do the lifecycles of formal parameters and instance variables 
	differ?}\\

Instance variables are persistent (limited by lifetime of the object) and
the lifetime of formal parameters is not really defined in runtime.

\subsubsection*{Exercise 26}
\textit{How would you translate the expressions "assignment" and 
	"expression" in german?}\\

\begin{itemize}
	\item assignment $=$ Zuweisung
	\item expression $=$ Ausdruck
\end{itemize}

\subsubsection*{Exercise 27}
\textit{How does an assignment-instruction work exactly? What's about to 
	be aware of in relation to data types?}\\

An assignment can be done with the operator "\lstinline{=}". For example:
\begin{lstlisting}
// create a instance variable for speed
private int speed;

// set the speed
public void setSpeed(int newSpeed)
{
	speed = newSpeed;
}
\end{lstlisting}
By assigning data
you have to be aware of data types. For example you can't assign a 
\lstinline{int} to a \lstinline{float} and so on. There are some strategies
to "cast" or "parse" data between different data types but that's not our 
topic now.

\newpage
\subsection{Team Exercise 1-4}
Create a Balloon-Class and create some objects and interact with them.
\lstinputlisting{../workspace/balloon/src/flight/Balloon.java}
\lstinputlisting{../workspace/balloon/src/flight/Main.java}

\newpage
\subsection{Team Exercise 5}
You want to write records, so you have to write a class Book for this.
This class shall have the following four attributes:
\begin{itemize}
	\item Title (String)
	\item Author (String)
	\item Price (float)
	\item Year on buy (int)
\end{itemize}

\noindent
The class shall also have two constructors.
\begin{itemize}
	\item Title and author are parameters. The books is not bought yet and
		this is why the price is 0.0 and the "year of buy" is -1.
	\item All attributes are initialized by parameters.
\end{itemize}

\noindent
The class shall have the following methods.
\begin{itemize}
	\item Two methods to get the title and author.
	\item A method to get and to set the year of buy.
	\item A method to get and to set the price.
\end{itemize}

\lstinputlisting{../workspace/Book/src/library/Book.java}
\lstinputlisting{../workspace/Book/src/library/Main.java}

\newpage
\subsection{Team Exercise 5 - Optional}
Think about banc accounts, their behaviour and attributes.
Implement a class Account. To avoid round sum problems work with
integer values. Play around with your class and get you some money!

\lstinputlisting{../workspace/Account/src/money/Account.java}
\lstinputlisting{../workspace/Account/src/money/Main.java}

\subsection{Summary exercises}

