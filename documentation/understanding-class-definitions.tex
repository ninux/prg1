\section{Understanding class definitions}

\subsection{Start with Eclipse}
In the first chapter we've worked with the BlueJ IDE but now I want to check
Java-Coding with a common and popular Java-IDE like Eclipse
To get the BlueJ-Projects work with Eclipse there are some things that have to
be done.

\begin{enumerate}
\item Create a new project in Eclipse.
\item Import the source (BlueJ example-code).
\item Add a package-name to the source.
\item Create a main (replaces all interaction which were invoked by hand).
\end{enumerate}

\lstinputlisting[caption=TicketMachine,firstline=12,lastline=20]{../workspace/naive-ticket-machine/src/foobar/TicketMachine.java}

\lstinputlisting[caption=Main (TicketMachine)]{../workspace/naive-ticket-machine/src/foobar/Main.java}

\newpage
\subsection{Chapter Exercises}
\subsubsection*{Exercise 2.21}
\textit{Suppose that the class Pet has a field called name that is of type
	String. Write an assignment statement in the body of the following 
	constructor so that the name field will be initialized with the value
	of the constructor's parameter.}\\

\begin{lstlisting}
public Pet(String petsName)
{
	name = petsName;
}
\end{lstlisting}

\subsubsection*{Exercise 2.22 (challenge)}
\textit{The following object creation will result in the constructor of the
	Date class being called. Can you write the constructor's header?\\
	\indent \lstinline{new Date("March", 23, 1861)}\\
	Try to give meaningful names to the paramenters.}\\

\begin{lstlisting}
public Date(String month, int day, int year)
{
	...
}
\end{lstlisting}

\newpage
\subsection{Selfstudy-Questions OOP2}
\subsubsection*{Exercise 4}
\textit{A class is build by three essential components. What are they?}\\

\subsubsection*{Exercise 5}
\textit{What is the order of the three components?}\\

\subsubsection*{Exercise 6}
\textit{What's their purpose?}\\

\subsubsection*{Exercise 8}
\textit{What is a variable?}\\

\subsubsection*{Exercise 9}
\textit{What are the synonyms to instance variables?}\\

\subsubsection*{Exercise 10}
\textit{What do you think where the term instance variable comes from?}\\

\subsubsection*{Exercise 11}
\textit{How can you put comments into a Java-Code?}\\

\subsubsection*{Exercise 12 (important)}
\textit{With which access-modification do you declare instance variables
	usually? Is it \lstinline{private} or \lstinline{public}? Do you
	have a reason for your answer?}\\

\subsubsection*{Exercise 13}
\textit{Explain the relation between a constructor and the state of an 
	onject.}\\

\subsubsection*{Exercise 14}
\textit{How do we name constructors?}\\

\subsubsection*{Exercise 15}
\textit{How long are the variables of an object alive (reachable)?}\\

\subsubsection*{Exercise 16}
\textit{Why sould you (if possible) initialise instance variables explicit?}\\

\subsubsection*{Exercise 17}
\textit{What's the defualt value which is given to a \lstinline{int} variavle
	by its initialisation?}\\

\subsubsection*{Exercise 19}
\textit{What's the use of parameters?}\\

\subsubsection*{Exercise 20}
\textit{What's the difference between a formal and a actual parameter?}\\


\subsection{Summary exercises}

