\subsection{Selfstudy-Questions OOP3}

\subsubsection*{Exercise 1}
\textit{What is a header? What is a body?}\\

\subsubsection*{Exercise 2}
\textit{Write down the signatures of the methods form class 
	TicketMachine}\\

\subsubsection*{Exercise 3}
\textit{Where can you place expressions and definitions?}\\

\subsubsection*{Exercise 4}
\textit{What is a block?}\\

\subsubsection*{Exercise 5}
\textit{How many \lstinline{return} expressions do you find in Code 2.1?}\\

\subsubsection*{Exercise 7}
\textit{What's the meaning of the return-type \lstinline{void}?}\\

\subsubsection*{Exercise 8}
\textit{Fill out the table.}\\

\begin{table}
	\centering
	\begin{tabular}{ll}
		\textbf{compound assignment} & \textbf{assignment} \\
		\lstinline!a += b! & \lstinline!a = a + b! \\
		\lstinline!a -= b! & \\
		\lstinline!a *= b! & \\
		\lstinline!a /= b! & \\
	\end{tabular}
\end{table}

\subsubsection*{Exercise 9}
\textit{In the code of the TicketMachine, there are two places where you
	can place a compound assignment operator. Find those two 
	places.}\\

\subsubsection*{Exercise 12}
\textit{Describe the conditional operator of the pseudo-code on page 42
	in german. Try to translate the code in german (except for the keywords
	\lstinline{if} and \lstinline{else}.}\\

\subsubsection*{Exercise 16}
\textit{At pitfall on page 48 is a very important information.
	Translate the first sentence in german.}\\

\subsubsection*{Exercise 17}
\textit{Fill out the following table.}\\

\begin{table}
	\centering
	\begin{tabular}{l | l | l | l}
			& \textbf{Field} 
				& \textbf{formal parameter} 
					& \textbf{local variable} \\
		\hline
		can store values? 
			&
				&
					& \\
		Where is/are they defined?
			&
				&
					& \\
		How long do they exist?
			&
				&
					& \\
		From where can you access them?
			&
				&
					& 
	\end{tabular}
\end{table}

