\subsection{Selfstudy-Questions OOP3}

\subsubsection*{Exercise 1}
\textit{What is a header? What is a body?}\\

\subsubsection*{Exercise 2}
\textit{Write down the signatures of the methods form class 
	TicketMachine}\\

\subsubsection*{Exercise 3}
\textit{Where can you place expressions and definitions?}\\

\subsubsection*{Exercise 4}
\textit{What is a block?}\\

\subsubsection*{Exercise 5}
\textit{How many \lstinline{return} expressions do you find in Code 2.1?}\\

\subsubsection*{Exercise 7}
\textit{What's the meaning of the return-type \lstinline{void}?}\\

\subsubsection*{Exercise 8}
\textit{Fill out the table.}\\

\begin{table}
	\centering
	\begin{tabular}{ll}
		\textbf{compound assignment} & \textbf{assignment} \\
		\lstinline!a += b! & \lstinline!a = a + b! \\
		\lstinline!a -= b! & \\
		\lstinline!a *= b! & \\
		\lstinline!a /= b! & \\
	\end{tabular}
\end{table}

\subsubsection*{Exercise 9}
\textit{In the code of the TicketMachine, there are two places where you
	can place a compound assignment operator. Find those two 
	places.}\\

\subsubsection*{Exercise 12}
\textit{Describe the conditional operator of the pseudo-code on page 42
	in german. Try to translate the code in german (except for the keywords
	\lstinline{if} and \lstinline{else}.}\\

\subsubsection*{Exercise 16}
\textit{At pitfall on page 48 is a very important information.
	Translate the first sentence in german.}\\

\subsubsection*{Exercise 17}
\textit{Fill out the following table.}\\

\begin{table}
	\centering
	\begin{tabular}{l | l | l | l}
			& \textbf{Field} 
				& \textbf{formal parameter} 
					& \textbf{local variable} \\
		\hline
		can store values? 
			&
				&
					& \\
		Where is/are they defined?
			&
				&
					& \\
		How long do they exist?
			&
				&
					& \\
		From where can you access them?
			&
				&
					& 
	\end{tabular}
\end{table}

\subsection{Exercises from book (chapter 2)}
\subsubsection*{Exercise 2.23}
\textit{Compare the header and body of the \lstinline?getBalance? method with 
the header and body of the \lstinline?getPrice? method. What are the 
differences between them? }

\subsubsection*{Exercise 2.24}
\textit{If a call to \lstinline?getPrice? can be characterized as 
"'What do ticketscost?"' how would you characterize a call to 
\lstinline?getBalance?? }

\subsubsection*{Exercise 2.25}
\textit{If the name of \lstinline?getBalance? is changed to 
\lstinline?getAmount?, does the return statement in the body of the method 
also need to be changed for the code to compile? Try it out with BlueJ. What 
does this tell you about the name of an accessor method and the nema of the 
field associated with it? }

\subsubsection*{Exercise 2.26}
\textit{Write an accessor method \lstinline?getTotal? in the 
\lstinline?TickatMachine? class. The new method should return the value of the 
\lstinline?total? field. }

\subsubsection*{Exercise 2.27}
\textit{Try removing the return statement from the body of 
\lstinline?getPrice?. What error message do you see now when you try compiling 
the class? }

\subsubsection*{Exercise 2.28}
\textit{Compare the method headers of \lstinline?getPrice? and 
\lstinline?printTicket? in Code 2.1. Apart from their names, what is the main 
difference between them? }

\subsubsection*{Exercise 2.29}
\textit{Do the \lstinline?insertMoney? and \lstinline?printTicket? methods have 
return statements? Why do you think this might be? Do you notice anything about 
their headers that might suggest why they do not require return statements? }

\subsubsection*{Exercise 2.30}
\textit{Create a ticket machine with a ticket price of your choosing. Before 
doing anything else, call the \lstinline?getBalance? method on it. Now call the 
\lstinline?insertMoney? method(Code2.6) and give an non-zero positive amount of 
money as the actual parameter. Now call \lstinline?getBalance? again. The two 
calls to \lstinline?getBalance? should show different outputs, because the call 
to \lstinline?insertMoney? had the effect of changing the machine's state via 
its \lstinline?balance? field. }

\subsubsection*{Exercise 2.31}
\textit{How can we tell from just its header that \lstinline?setPrice? is a 
method and not a constructor? \\
\lstinline?public void setPrice(int cost)?}

\subsubsection*{Exercise 2.32}
\textit{Complete the body of the \lstinline?setPrice? method so that it 
assigns the value of its parameter to the \lstinline?price? field. }

\subsubsection*{Exercise 2.33}
\textit{Complete the body of the following method, whose purpose is to add the 
value of its parameter to a field named \lstinline?score?. \\
\lstinline?/**
 * Increase score by the given number of points. 
 */
public void increase(int points)
{
    ...
}?}

\subsubsection*{Exercise 2.34}
\textit{Is the increase method a mutator? If so, how could you demonstrate 
this? }

\subsubsection*{Exercise 2.35}
\textit{Complete the following method, whose purpose is to subtract the value 
of its parameter from a field named \lstinline?price?. \\
\lstinline?/**
 * Reduce price by the given amount. 
 */
public void discount(int amount)
{
    ...
}?}

\subsubsection*{Exercise 2.36}
\textit{Write down exactly what will be printed by the following statement: \\
\lstinline?System.out.println("My cat has green eyes.");?}

\subsubsection*{Exercise 2.37}
\textit{Add a method called \lstinline?prompt? to the \lstinline?TicketMachine? 
class. This should have a void return type and take no parameters. The body of 
the method should print the following single line of output: \\
\lstinline?Please insert the correct amount of money. ?}

\subsubsection*{Exercise 2.38}
\textit{What do you think would be printed if you altered the fourth statement 
of \lstinline?printTicket? so that \lstinline?price? also has quotes around it, 
as follows? \\
\lstinline?System.out.println("\# " + "price" + " cents.");?}

\subsubsection*{Exercise 2.39}
\textit{What about the following version? \\
\lstinline?SSystem.out.println("\# price cents.");?}

\subsubsection*{Exercise 2.40}
\textit{Could either of the previous two versions be used to show the price of 
tickets in different ticket machines? Explain your answer. }

\subsubsection*{Exercise 2.41}
\textit{Add a \lstinline?showPrice? method to the \lstinline?TicketMachine? 
class. This should have a void return type and take no parameters. The body of 
the method should print: \\
\lstinline?The price of a ticket is xyz cents- ?}

\subsubsection*{Exercise 2.42}
\textit{Create two ticket machines with differently priced tickets. Do calls 
to their \lstinline?showPrice? methods show the same output, or different? How 
do you explain this effect? }

\subsubsection*{Exercise 2.43}
\textit{Modify the constructor of \lstinline?TicketMachine? so that it no 
longer has a parameter. Instead, the price of tickets should be fixed at 1,000 
cents. What effect does this have when you construct ticket.machine objects 
within BlueJ? }

\subsubsection*{Exercise 2.44}
\textit{Give the class two constructors. One should take a single parameter 
that specifies the price, and the other should take no parameter and set the 
price to be a default value of your choosing. Test your implementation by 
creating machines via two different constructors. }

\subsubsection*{Exercise 2.45}
\textit{Implement a method, \lstinline?empty?, that simulates the effect of 
removing all money from the machine. This method should have a void return 
type, and its body should simply set the \lstinline?total? field to zero. Does 
this method need to take any parameters? Test your method by reating a 
machine, inserting some moneym printing some tickets, checking the total, and 
then emptying the machine. Is the \lstinline?empty? method a mutator or an 
accessor? }

\subsubsection*{Exercise 2.46}
\textit{Check that the behavior discussed here is accurate by creating a 
\lstinline?TicketMachine? instance and calling \lstinline?insertMoney? with 
various actual parameter values. Check the balance both before and after 
calling \lstinline?insertMoney?. Does the balance ever change in the cases 
when an error message is printed? Try to predict what will happen if you enter 
the value zero as the parameter, and then see if you are right. }

\subsubsection*{Exercise 2.47}
\textit{Predict what you think will happen if you change the test in 
\lstinline?insertMoney? to use the greater-than or equal-to operator: \\
\lstinline?if(amount >= 0)?\\
Check your predictions by running some tests. What difference does it make to 
the behavior of the method? }

\subsubsection*{Exercise 2.48}
\textit{Rewrite the if-else statement so that the error message is printed if 
the boolean expression is true but the balance is increased if the expression 
is false. You will obviously have to rewrite the condition to make things 
happen this way arround. }

\subsubsection*{Exercise 2.49}
\textit{In the figures project we looked at in Chapter 1 we used a 
\lstinline?boolean? field to control a feature of the circle objects. What was 
that feature? Was it well suited to being controlled by a type with only two 
different values? }

\subsubsection*{Exercise 2.50}
\textit{In this version of \lstinline{printTicket}, we also do something 
slightly different with the \lstinline{total} and \lstinline{balance} fields. 
Compare the implementation of the method in Code 2.1 with that in Code 2.8 to 
see whether you can tell what those differences are. Then check your 
understanding by experimenting within BlueJ. }

\subsubsection*{Exercise 2.51}
\textit{Is it possible to remove the else part of the statement in the 
\lstinline{printTicket} method (i.e., remove the word \lstinline{else} and the 
block attached to it)? Try doing this and seeing if the code still compiles. 
What happens now if you try to print a ticket without inserting any money? }

\subsubsection*{Exercise 2.52}
\textit{After a ticket has been printed, could the value in the 
\lstinline{balance} field ever be set to a negative value by subtracting 
\lstinline{price} from it? Justify your answer. }

\subsubsection*{Exercise 2.53}
\textit{So far, we have introduced you to two arithmetic operators, + and -, 
that can be used in \textbf{arithmetic expressions} in Java. Take a look at 
Appendix C to find out what other operators are available. }

\subsubsection*{Exercise 2.54}
\textit{Write an assignment statement that will store the result of 
multiplying two variables, \lstinline{price} and \lstinline{discount}, into a 
third variable, \lstinline{saving}}

\subsubsection*{Exercise 2.55}
\textit{Write an assignment statement that will divide the value in 
\lstinline{total} by the value in \lstinline{count} and store the result in 
\lstinline{mean}. }

\subsubsection*{Exercise 2.56}
\textit{Write an if statement that will compare the value in \lstinline{price} 
against the value in \lstinline{budget}. If lstinline{price} is greater than 
\lstinline{budget}, then print the message "'Too expensive"'; otherwise print 
the message "'Just right"'. }

\subsubsection*{Exercise 2.57}
\textit{Modify your answer to the previous exercise so that the message 
includes the value of your budget if the price is too high. }

