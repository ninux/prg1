\subsection{Selfstudy-Questions OOP9}

\subsubsection{Chapter 6.1 to 6.4 - Code duplication}

\subsubsection*{Exercise }1
\textit{Solve the exercises 6.1, 6.2, 6.4 and 6.5} \\

\subsubsection*{Exercise 2}
\textit{What is coupling and how should it be?} \\

\subsubsection*{Exercise 3}
\textit{What is cohesion and how should it be?} \\

\subsubsection*{Exercise 4}
\textit{What's the problem with code duplication?} \\

\subsubsection*{Exercise 5}
\textit{Code duplication is a symptom for what? Bad coupling or bad cohesion? } \\

\subsubsection{Chapter 6.5 to 6.11 - Cohesion}

\subsubsection*{Exercise 6}
\textit{Solve the exercises 6.6 to 6.8, 6.11, 6.14, 6.16 and 6.17} \\

\subsubsection*{Exercise 7}
\textit{The Room class is now managing neighboring rooms with a HashMap.
To do so the program had to be changed at a lot of different places. 
Is this an evidence of strong or weak coupling?} \\

\subsubsection*{Exercise 8}
\textit{What is information hiding or encapsulation and what's the effect
of this fundamental principle?} \\

\subsubsection*{Exercise 9}
\textit{What's the meaning of the concept "localizing change"?} \\

\subsubsection*{Exercise 10}
\textit{What's the difference between explicit and implicit coupling?} \\

\subsubsection*{Exercise 11}
\textit{Do cohesive methods make also sense?} \\

\subsubsection*{Exercise 12}
\textit{What are the two most powerful benefits of strong cohesion?} \\

\subsubsection{Chapter 6.12 to 6.14 - Design guidelines}

\subsubsection*{Exercise 13}
\textit{Solve the exercise 6.27} \\

\subsubsection*{Exercise 14}
\textit{What's the reason for refactoring?} \\

\subsubsection*{Exercise 15}
\textit{Describe the method or the steps at a refactoring.} \\

\subsubsection*{Exercise 16}
\textit{At which point is a method too long?} \\

\subsubsection*{Exercise 17}
\textit{At which point is a class too complex?} \\

