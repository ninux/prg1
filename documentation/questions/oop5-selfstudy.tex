\subsection{Selfstudy-Questions OOP5}

\subsubsection{Chapter 4.1 to 4.3 - An organizer for music files}

\subsubsection*{Exercise 1}
\textit{Solve the exercises 4.1 to 4.3}\\

\subsubsection*{Exercise 2}
\textit{What do you understand by "Java-Package"?}\\

\subsubsection*{Exercise 3}
\textit{You want to use the library-class ArrayList. What expression makes it
possible to use that library-class in your source code?}\\

\subsubsection{Chapter 4.4 to 4.7 - Numbering within collections}

\subsubsection*{Exercise 4}
\textit{Solve the exercises 4.4 to 4.7}\\

\subsubsection*{Exercise 5}
\textit{Solve the exercises 4.8 to 4.11}\\

\subsubsection*{Exercise 6}
\textit{Solve the exercises 4.12 to 4.13}\\

\subsubsection*{Exercise 7}
\textit{Explanin the following declaration:}\\
\lstinline?private ArrayList<Balloon> list = new ArrayList<>();?\\

\subsubsection*{Exercise 8}
\textit{What is the connection between abstraction an ArrayLists?}\\

\subsubsection*{Exercise 9}
\textit{What is the difference of the methods remove() and get() on
ArrayLists?}\\

\subsubsection{Chapter 4.8 to 4.12 - The Iterator type}

\subsubsection*{Exercise 10}
\textit{Solve the exercises 4.18 to 4.19}\\

\subsubsection*{Exercise 11}
\textit{Solve the exercise 4.22}\\

\subsubsection*{Exercise 12}
\textit{Explain as detailed as possible the source code on page 108.}\\

\subsubsection*{Exercise 13}
\textit{Is it possible, that the body of an while-loop is never executed?}\\

\subsubsection*{Exercise 14}
\textit{Show two alternative expressions for no++}\\

\subsubsection*{Exercise 15}
\textit{An ArrayList can be traversed by an foreach-loop. Do you know other
ways to do the same?}\\

\subsubsection*{Exercise 16}
\textit{Is hasNext() a method of ArrayList or Iterator? How do you have to 
understand/interpret the return-value of hasNext()?}\\

\subsubsection{Chapter 4.14 - Summary of the music-organizer project}

\subsubsection*{Exercise 17}
\textit{DO NOT READ THIS CHAPTER, JUST READ THE CONCEPT-BOX AT PAGE 130.}\\

\subsubsection*{Exercise 18}
\textit{A variable that is declared for a classtype (or so called 
reference-variable) can store the special value null. Explain the situation 
with a drwing/sektch. What does it look like, if it's storing an object?}\\

\subsubsection{Chapter 4.15 to 4.17 - Summary}

\subsubsection*{Exercise 19}
\textit{Solve the exercises 4.62 to 4.65}\\

\subsubsection*{Exercise 20}
\textit{Solve the exercises 4.66 to 4.68}\\

\subsubsection*{Exercise 21}
\textit{What are the pros and cons of Arrays?}\\

\subsubsection*{Exercise 22}
\textit{How do you get the length of an Array?}\\

\subsubsection*{Exercise 23}
\textit{Solve the exercises 4.69, 4.71, 4.73 and 4.74}\\




