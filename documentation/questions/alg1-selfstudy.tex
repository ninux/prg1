\subsection{Selfstudy-Questions ALG1}

\subsubsection*{Exercise 10}
\textit{Describe in words or as pseudocode one algorithm per problem.}
\begin{enumerate}[label={(\alph*)}]
	\item Calculate the product of two integers without a multiplication-operator.
	\item Find the lowest number out of a sequence like $S = {4,-1,50,10,0,1,-2,5,10}$
\end{enumerate}

\subsubsection*{Exercise 11}
\textit{Implement an algorithm to calculate the greatest common divisor
with the modulo operator (see ALG1 presentation, page 12).}
\begin{enumerate}[label={(\alph*)}]
	\item Test your algorithm with some examples.
	\item Change your algorithm so that you don't use the modulo operator. 
		Change the modulo operation with a combined expression and test 
		your implementation again. \\
		\textit{Hint: The modulo operator can be substuituted by a 
			sequence of subtractions that is finished as the 
			result is smaller than the subtrahend.}
\end{enumerate}

\subsubsection*{Exercise 12}
\textit{What's the order of the algorithm to calculate the n-th pseudo random 
number z (see OOP6 presentation page 28 and ALG1 presentation page 15)?}
\[ z_{n+1} = (a \cdot Z_n + r) \% m \]
