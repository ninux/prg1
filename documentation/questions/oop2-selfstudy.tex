\subsection{Selfstudy-Questions OOP2}
\subsubsection*{Exercise 4}
\textit{A class is build by three essential components. What are they?}\\

\subsubsection*{Exercise 5}
\textit{What is the order of the three components?}\\

\subsubsection*{Exercise 6}
\textit{What's their purpose?}\\

\subsubsection*{Exercise 8}
\textit{What is a variable?}\\

\subsubsection*{Exercise 9}
\textit{What are the synonyms to instance variables?}\\

\subsubsection*{Exercise 10}
\textit{What do you think where the term instance variable comes from?}\\

\subsubsection*{Exercise 11}
\textit{How can you put comments into a Java-Code?}\\

\subsubsection*{Exercise 12 (important)}
\textit{With which access-modification do you declare instance variables
	usually? Is it \lstinline{private} or \lstinline{public}? Do you
	have a reason for your answer?}\\

\subsubsection*{Exercise 13}
\textit{Explain the relation between a constructor and the state of an 
	onject.}\\

\subsubsection*{Exercise 14}
\textit{How do we name constructors?}\\

\subsubsection*{Exercise 15}
\textit{How long are the variables of an object alive (reachable)?}\\

\subsubsection*{Exercise 16}
\textit{Why sould you (if possible) initialise instance variables explicit?}\\

\subsubsection*{Exercise 17}
\textit{What's the defualt value which is given to a \lstinline{int} variavle
	by its initialisation?}\\

\subsubsection*{Exercise 19}
\textit{What's the use of parameters?}\\

\subsubsection*{Exercise 20}
\textit{What's the difference between a formal and a actual parameter?}\\

\subsubsection*{Exercise 21}
\textit{Is the following statement correct; "formal parameters are special
	variables"?}\\

\subsubsection*{Exercise 22}
\textit{What's about the accessability of formal parameters?}\\

\subsubsection*{Exercise 23}
\textit{In which way this differs from instance variables?}\\

\subsubsection*{Exercise 24}
\textit{How do the lifecycles of formal parameters and instance variables 
	differ?}\\

\subsubsection*{Exercise 26}
\textit{How would you translate the expressions "assignment" and 
	"expression"?}\\

\subsubsection*{Exercise 27}
\textit{How does an assignment-instruction work exactly? What's about to 
	be aware of in relation to data types?}\\

\subsection{Exercises from book (chapter 2)}
\subsubsection{Exercise 2.1}
Create a \verb?TicketMachine? object bench and take a look at its methods. You 
should see the following: \verb?get Balance, getPrice, insertMoney,? and 
\verb?ptintTicket?. Try out the \verb?getPrice? method. You should see a 
return value containing the price of the ticket thet was set when this object 
was created. Use the \verb?insertMoney? method to simulate inserting an amount 
of money into the machine. The machine stores as a balance the amount of money 
inserted. Use \verb?getBalance? to check that the machine has kept an accurate 
record of the amount just inserted. You can insert several seperate amounts of 
money into the machine, just like you might insert multiple coins or bills into 
a real machine. Try inserting the exact amount required for a ticket, and use 
\verb?getBalance? to ensure that the balance is increased correctly. As this is 
a simple machine, a ticket will not be issued automatically, so once you have 
inserted enough money, call the \verb?printTicket? method. A facsimile ticket 
should be printed in the BlueJ terminal window

\subsubsection{Exercise 2.2}
What value is returned if you get the machine's balance after it has printed a 
ticket? 

\subsubsection{Exercise 2.3}
Experiment with inserting different amounts of money before printing tickets. 
Do you notice anything strange about the machine's behavior? what happens if 
you insert too much money into the machine - do you receive any refund? what 
happens if you do not insert enough and then try to print a ticket? 

\subsubsection{Exercise 2.4}
Try to obtain a good understanding of the ticket machine's behavior by 
interacting with it on the object bench before we start looking, in the next 
section, at how the \verb?TicketMachine? class is implemented. 

