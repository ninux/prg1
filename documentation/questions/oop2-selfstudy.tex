\subsection{Selfstudy-Questions OOP2}
\subsubsection*{Exercise 4}
\textit{A class is build by three essential components. What are they?}\\

\subsubsection*{Exercise 5}
\textit{What is the order of the three components?}\\

\subsubsection*{Exercise 6}
\textit{What's their purpose?}\\

\subsubsection*{Exercise 8}
\textit{What is a variable?}\\

\subsubsection*{Exercise 9}
\textit{What are the synonyms to instance variables?}\\

\subsubsection*{Exercise 10}
\textit{What do you think where the term instance variable comes from?}\\

\subsubsection*{Exercise 11}
\textit{How can you put comments into a Java-Code?}\\

\subsubsection*{Exercise 12 (important)}
\textit{With which access-modification do you declare instance variables
	usually? Is it \lstinline{private} or \lstinline{public}? Do you
	have a reason for your answer?}\\

\subsubsection*{Exercise 13}
\textit{Explain the relation between a constructor and the state of an 
	object.}\\

\subsubsection*{Exercise 14}
\textit{How do we name constructors?}\\

\subsubsection*{Exercise 15}
\textit{How long are the variables of an object alive (reachable)?}\\

\subsubsection*{Exercise 16}
\textit{Why sould you (if possible) initialise instance variables explicit?}\\

\subsubsection*{Exercise 17}
\textit{What's the defualt value which is given to a \lstinline{int} variavle
	by its initialisation?}\\

\subsubsection*{Exercise 19}
\textit{What's the use of parameters?}\\

\subsubsection*{Exercise 20}
\textit{What's the difference between a formal and a actual parameter?}\\

\subsubsection*{Exercise 21}
\textit{Is the following statement correct; "formal parameters are special
	variables"?}\\

\subsubsection*{Exercise 22}
\textit{What's about the accessability of formal parameters?}\\

\subsubsection*{Exercise 23}
\textit{In which way this differs from instance variables?}\\

\subsubsection*{Exercise 24}
\textit{How do the lifecycles of formal parameters and instance variables 
	differ?}\\

\subsubsection*{Exercise 26}
\textit{How would you translate the expressions "assignment" and 
	"expression"?}\\

\subsubsection*{Exercise 27}
\textit{How does an assignment-instruction work exactly? What's about to 
	be aware of in relation to data types?}\\

\subsection{Exercises from book (chapter 2)}
\subsubsection*{Exercise 2.1}
\textit{Create a \lstinline{TicketMachine} object bench and take a look at its 
methods. You should see the following: 
\lstinline{get Balance, getPrice, insertMoney,} and \lstinline{printTicket}. Try out the 
\lstinline{getPrice} method. You should see a return value containing the price of 
the ticket thet was set when this object was created. Use the 
\lstinline{insertMoney} method to simulate inserting an amount of money into the 
machine. The machine stores as a balance the amount of money inserted. Use 
\lstinline{getBalance} to check that the machine has kept an accurate record of the 
amount just inserted. You can insert several seperate amounts of money into the 
machine, just like you might insert multiple coins or bills into a real 
machine. Try inserting the exact amount required for a ticket, and use 
\lstinline{getBalance} to ensure that the balance is increased correctly. As this 
is a simple machine, a ticket will not be issued automatically, so once you 
have inserted enough money, call the \lstinline{printTicket} method. A facsimile 
ticket should be printed in the BlueJ terminal window. }

\subsubsection*{Exercise 2.2}
\textit{What value is returned if you get the machine's balance after it has 
printed a ticket? }

\subsubsection*{Exercise 2.3}
\textit{Experiment with inserting different amounts of money before printing 
tickets. Do you notice anything strange about the machine's behavior? what 
happens if you insert too much money into the machine - do you receive any 
refund? what happens if you do not insert enough and then try to print a 
ticket? }

\subsubsection*{Exercise 2.4}
\textit{Try to obtain a good understanding of the ticket machine's behavior by 
interacting with it on the object bench before we start looking, in the next 
section, at how the \lstinline{TicketMachine} class is implemented. }

\subsubsection*{Exercise 2.5}
\textit{Create another ticket machine for tickets of a different price; 
remember that you have to supply this value when you create the machine object. 
Buy a ticket from that machine. Does the printed ticket look any different 
from those printed by the first machine? }

\subsubsection*{Exercise 2.6}
\textit{Write out what you think the outer wrappers of the \lstinline{Student} and 
\lstinline{LabClass} classes look like; do not worry about the inner part. }

\subsubsection*{Exercise 2.7}
\textit{Does it matter whether we write \lstinline{public class TicketMachine} or 
\lstinline{class public TicketMachine} in the outer wrapper of the class? Edit the 
source of the \lstinline{TicketMachine} class to make the change, and then close 
the editor window. Do you notice a change in the class diagram? }

\subsubsection*{Exercise 2.8}
\textit{Check whether or not it is possible to leave out the word \lstinline{public} 
from the outer wrapper of the \lstinline{TicketMachine} class. }

\subsubsection*{Exercise 2.9}
\textit{Put back the word \lstinline{public}, and then check whether it is possible 
to leave out the word \lstinline{class} by trying to compile again. Make sure that 
both words are put back as they were originally before continuing. }

\subsubsection*{Exercise 2.10}
\textit{From your earlier experimentation with the ticket machine objects 
within BlueJ, you can probably remember the names of some of the methods-
\lstinline{printTicket}, for instance. Look at the class definition in Code 2.1, 
and use this knowledge, along with the additional information about ordering 
we have given you, to make a list of the names of the fields, constructors and 
methods in the \lstinline{TicketMachine} class. Hint: There is only one constructor 
in the class. }

\subsubsection*{Exercise 2.11}
\textit{What are the two features of the constructor that make it look 
signifigantly different from the methods of the class? }

\subsubsection*{Exercise 2.12}
\textit{What do you think is the type of each of the following fields? \\
\lstinline{private int count;}\\
\lstinline{private Student representative;}\\
\lstinline{private Game game;}}

\subsubsection*{Exercise 2.13}
\textit{What are the names of the following fields? \\
\lstinline{private boolean alive;}\\
\lstinline{private Person tutor;}\\
\lstinline{private Game game;}}

\subsubsection*{Exercise 2.14}
\textit{From what you know about the naming conventions for classes, which of 
the type names in Exercises 2.12 and 2.13 would you say are class names? }

\subsubsection*{Exercise 2.15}
\textit{In the following fiels declaration from the \lstinline{TicketMachine} class\\
\lstinline{private int price;}\\
does it matter which order the three words appear in? Edit the 
\lstinline{TicketMachine} class to try different orderings. After each change, 
close the editor. Does the appearance of the class diagram after each change 
give you a clue as to whether or not other orderings are possible? Check by 
pressing the Compile button to see if there is an error message. }

\subsubsection*{Exercise 2.16}
\textit{Is it always necessary to have a semicolon at the end of a field 
declaration? Once again, expreiment via the editor. The rule you will learn 
here is an important one, so be sure to remember it. }

\subsubsection*{Exercise 2.17}
\textit{Write in full the declaration for a field of type \lstinline{int} whose 
name is \lstinline{status}. }

\subsubsection*{Exercise 2.18}
\textit{To what class does the following constructor belong? \\
\lstinline{public Student(String name)}}

\subsubsection*{Exercise 2.19}
\textit{How many parameters does the following constructor have, and what are 
their types? \\
\lstinline{public Book(String title, double, price)}}

\subsubsection*{Exercise 2.20}
\textit{Can you guess what types some of the \lstinline{Book} class's fields might 
be, from the parameters in its constructor? Can you assume anything about the 
names of the fields? }

\subsubsection*{Exercise 2.21}
\textit{Suppose that the class \lstinline{Pet} has a field called \lstinline{name} that 
is of type \lstinline{String}. Write an assignment in the body of the following 
constructor so that the \lstinline{name} field will be initialized with the value 
of the constructor's parameter. \\
\lstinline{public Pet(String petsName)}\\
\lstinline!\{!\\
\lstinline!\}!}

\subsubsection*{Exercise 2.22}
\textit{Challange exercise The following object creation will result in the 
constructor of the \lstinline{Date} class being called. Can you write the 
constructor's header? \\
\lstinline{new Date("March", 23, 1861)}\\
Try give meaningful names to the parameters. }

\subsubsection*{Exercise 2.23}
\textit{Compare the header and body of the \lstinline?getBalance? method with the 
header and body of the \lstinline?getPrice? method. What are the differences 
between them? }

\subsubsection*{Exercise 2.24}
\textit{If a call to \lstinline?getPrice? can be characterized as 
"'What do ticketscost?"' how would you characterize a call to 
\lstinline?getBalance?? }

\subsubsection*{Exercise 2.25}
\textit{If the name of \lstinline?getBalance? is changed to \lstinline?getAmount?, does 
the return statement in the body of the method also need to be changed for the 
code to compile? Try it out with BlueJ. What does this tell you about the name 
of an accessor method and the nema of the field associated with it? }

\subsubsection*{Exercise 2.26}
\textit{Write an accessor method \lstinline?getTotal? in the \lstinline?TickatMachine? 
class. The new method should return the value of the \lstinline?total? field. }

\subsubsection*{Exercise 2.27}
\textit{Try removing the return statement from the body of \lstinline?getPrice?. 
What error message do you see now when you try compiling the class? }

\subsubsection*{Exercise 2.28}
\textit{Compare the method headers of \lstinline?getPrice? and \lstinline?printTicket? in 
Code 2.1. Apart from their names, what is the main difference between them? }

\subsubsection*{Exercise 2.29}
\textit{Do the \lstinline?insertMoney? and \lstinline?printTicket? methods have return 
statements? Why do you think this might be? Do you notice anything about their 
headers that might suggest why they do not require return statements? }

\subsubsection*{Exercise 2.30}
\textit{Create a ticket machine with a ticket price of your choosing. Before 
doing anything else, call the \lstinline?getBalance? method on it. Now call the 
\lstinline?insertMoney? method(Code2.6) and give an non-zero positive amount of 
money as the actual parameter. Now call \lstinline?getBalance? again. The two calls 
to \lstinline?getBalance? should show different outputs, because the call to 
\lstinline?insertMoney? had the effect of changing the machine's state via its 
\lstinline?balance? field. }

\subsubsection*{Exercise 2.31}
\textit{How can we tell from just its header that \lstinline?setPrice? is a method 
and not a constructor? \\
\lstinline?public void setPrice(int cost)?}

\subsubsection*{Exercise 2.32}
\textit{Complete the body of the \lstinline?setPrice? method so that it assigns the 
value of its parameter to the \lstinline?price? field. }

\subsubsection*{Exercise 2.33}
\textit{Complete the body of the following method, whose purpose is to add the 
value of its parameter to a field named \lstinline?score?. \\
\lstinline?/**
 * Increase score by the given number of points. 
 */
public void increase(int points)
{
    ...
}?}

\subsubsection*{Exercise 2.34}
\textit{Is the increase method a mutator? If so, how could you demonstrate 
this? }

\subsubsection*{Exercise 2.35}
\textit{Complete the following method, whose purpose is to subtract the value 
of its parameter from a field named \lstinline?price?. \\
\lstinline?/**
 * Reduce price by the given amount. 
 */
public void discount(int amount)
{
    ...
}?}

\subsubsection*{Exercise 2.36}
\textit{Write down exactly what will be printed by the following statement: \\
\lstinline?System.out.println("My cat has green eyes.");?}

\subsubsection*{Exercise 2.37}
\textit{Add a method called \lstinline?prompt? to the \lstinline?TicketMachine? class. 
This should have a void return type and take no parameters. The body of the 
method should print the following single line of outpur: \\
\lstinline?Please insert the correct amount of money. ?}

\subsubsection*{Exercise 2.38}
\textit{What do you think would be printed if you altered the fourth statement 
of \lstinline?printTicket? so that \lstinline?price? also has quotes around it, as 
follows? \\
\lstinline?System.out.println("\# " + "price" + " cents.");?}

\subsubsection*{Exercise 2.39}
\textit{What about the following version? \\
\lstinline?SSystem.out.println("\# price cents.");?}

\subsubsection*{Exercise 2.40}
\textit{Could either of the previous two versions be used to show the price of 
tickets in different ticket machines? Explain your answer. }

\subsubsection*{Exercise 2.41}
\textit{Add a \lstinline?showPrice? method to the \lstinline?TicketMachine? class. This 
should have a void return type and take no parameters. The body of the method 
should print: \\
\lstinline?The price of a ticket is xyz cents- ?}

\subsubsection*{Exercise 2.42}
\textit{Create two ticket machines with differently priced tickets. Do calls 
to their \lstinline?showPrice? methods show the same output, or different? How do 
you explain this effect? }

\subsubsection*{Exercise 2.43}
\textit{Modify the constructor of \lstinline?TicketMachine? so that it no longer 
has a parameter. Instead, the price of tickets should be fixed at 1,000 cents. 
What effect does this have when you construct ticket.machine objects within 
BlueJ? }

\subsubsection*{Exercise 2.44}
\textit{Give the class two constructors. One should take a single parameter 
that specifies the price, and the other should take no parameter and set the 
price to be a default value of your choosing. Test your implementation by 
creating machines via two different constructors. }

\subsubsection*{Exercise 2.45}
\textit{Implement a method, \lstinline?empty?, that simulates the effect of 
removing all money from the machine. This method should have a void return 
type, and its body should simply set the \lstinline?total? field to zero. Does 
this method need to take any parameters? Test your method by reating a 
machine, inserting some moneym printing some tickets, checking the total, and 
then emptying the machine. Is the \lstinline?empty? method a mutator or an 
accessor? }

\subsubsection*{Exercise 2.46}
\textit{Check that the behavior discussed here is accurate by creating a 
\lstinline?TicketMachine? instance and calling \lstinline?insertMoney? with various 
actual parameter values. Check the balance both before and after calling 
\lstinline?insertMoney?. Does the balance ever change in the cases when an error 
message is printed? Try to predict what will happen if you enter the value 
zero as the parameter, and then see if you are right. }

\subsubsection*{Exercise 2.47}
\textit{Predict what you think will happen if you change the test in 
\lstinline?insertMoney? to use the greater-than or equal-to operator: \\
\lstinline?if(amount >= 0)?\\
Check your predictions by running some tests. What difference does it make to 
the behavior of the method? }

\subsubsection*{Exercise 2.48}
\textit{Rewrite the if-else statement so that the error message is printed if 
the boolean expression is true but the balance is increased if the expression 
is false. You will obviously have to rewrite the condition to make things 
happen this way arround. }

\subsubsection*{Exercise 2.49}
\textit{In the figures project we looked at in Chapter 1 we used a 
\lstinline?boolean? field to control a feature of the circle objects. What was that 
feature? Was it well suited to being controlled by a type with only two 
different values? }

