\subsection{Selfstudy-Questions OOP2}
\subsubsection*{Exercise 4}
\textit{A class is build by three essential components. What are they?}\\

\subsubsection*{Exercise 5}
\textit{What is the order of the three components?}\\

\subsubsection*{Exercise 6}
\textit{What's their purpose?}\\

\subsubsection*{Exercise 8}
\textit{What is a variable?}\\

\subsubsection*{Exercise 9}
\textit{What are the synonyms to instance variables?}\\

\subsubsection*{Exercise 10}
\textit{What do you think where the term instance variable comes from?}\\

\subsubsection*{Exercise 11}
\textit{How can you put comments into a Java-Code?}\\

\subsubsection*{Exercise 12 (important)}
\textit{With which access-modification do you declare instance variables
	usually? Is it \lstinline{private} or \lstinline{public}? Do you
	have a reason for your answer?}\\

\subsubsection*{Exercise 13}
\textit{Explain the relation between a constructor and the state of an 
	onject.}\\

\subsubsection*{Exercise 14}
\textit{How do we name constructors?}\\

\subsubsection*{Exercise 15}
\textit{How long are the variables of an object alive (reachable)?}\\

\subsubsection*{Exercise 16}
\textit{Why sould you (if possible) initialise instance variables explicit?}\\

\subsubsection*{Exercise 17}
\textit{What's the defualt value which is given to a \lstinline{int} variavle
	by its initialisation?}\\

\subsubsection*{Exercise 19}
\textit{What's the use of parameters?}\\

\subsubsection*{Exercise 20}
\textit{What's the difference between a formal and a actual parameter?}\\
