\newglossaryentry{state}
{
	name=state,
	description={A object or its status is represented by his state.
		The state is represented by the values in the fields
		(instance variables)},
	sort=state
}


\newglossaryentry{class}
{
	name=class,
	description={A class describes the kind of an object. This is done
		by giving instance variables and methods. The objects 
		represents individual instatiations of the class},
	plural=classes,
	sort=class
}

\newglossaryentry{object}
{
	name=object,
	description={An object is a instance of a class},
	plural=objects,
	sort=object
}

\newglossaryentry{method}
{
	name=method,
	description={A method is a action (function) of a specific class that
		can be invoked on an object of the given class. Objects usually
		do something when a method is invoked, so a good keyword to it
		would be \textit{what}, as most methods are named by a verb.
		The methods give the objects their own particular and
		characteristic behavior},
	plural=methods,
	sort=method
}

\newglossaryentry{parameter}
{
	name=parameter,
	description={Addition information (data) given to a method or object
		is called parameter},
	plural=parameters,
	sort=parameter
}

\newglossaryentry{signature}
{
	name=signature,
	description={The signature of a method is the part that identifies 
		it to the compiler. For example the signature of 
		\lstinline{public setSpeed(int newSpeed, int newTolerance)}
		is not the whole head of the method but the name 
		\lstinline{setSpeed} and the list of parameter-types 
		\lstinline{int ..., int ...}},
	plural=signatures,
	sort=signature
}

\newglossaryentry{type}
{
	name=type,
	description={The type defines the kind of data or value (for example 
		to a parameter, return value (see data types) or a variable},
	plural=types,
	sort=type
}

\newglossaryentry{instance}
{
	name=instance,
	description={An instance is a realisation of a class to a real object,
		so instance is a synonym to object},
	plural=instances,
	sort=instance
}

\newglossaryentry{field}
{
	name=field,
	description={Fields store data for an object. Fields are also
		known as instance or member variables},
	plural=fields,
	sort=field
}

\newglossaryentry{constructor}
{
	name=constructor,
	description={A constructor is a special method in a class which is
		responsible to initialize objects properly. In difference to
		usual methods it has no return value and is only used once},
	plural=constructors,
	sort=constructor
}

\newglossaryentry{accessor}
{
	name=accessor,
	description={An accessor or accessor method is a method that provides
		access to information about an object's state (get-methods)},
	plural=accessors,
	sort=accessor
}

\newglossaryentry{header}
{
	name=header,
	description={A header is a part of a method. It is the part that is
		not only including the signature but the whole definition.
		Example: \lstinline{public int getAge(String name)} is the
		header whereas \lstinline{getAge(String )} is the signature},
	plural=headers,
	sort=header
}

\newglossaryentry{scope}
{
	name=scope,
	description={The scope of a variable defines the section of source
		code from which the variable can be accessed},
	plural=scopes,
	sort=scope
}

\newglossaryentry{lifetime}
{
	name=lifetime,
	description={The lifetime of a variable describes how long the
		variable continues to exist before it is destroyed},
	plural=lifetimes,
	sort=lifetime
}

\newglossaryentry{assignment}
{
	name=assignment,
	description={An assignment (statement) is a directive to assign a
		value into a variable, for example 
		\lstinline{speed = newSpeed;} is an assignment},
	plural=assignments,
	sort=assignment
}

\newglossaryentry{body}
{
	name=body,
	description={A body is a part of an method. It is the part that is
		bordered by the curly braces. The whole content between these
		braces is called body (see header for contrast)},
	plural=bodies,
	sort=body
}

\newglossaryentry{mutator}
{
	name=mutator,
	description={A mutator or mutator method is a method that provides
		the ability to change fields of an object. For example
		\lstinline?changeSize(int newSize)? is a typical mutator
		method},
	plural=mutators,
	sort=mutator
}

\newglossaryentry{conditional statement}
{
	name=conditional statement,
	description={A conditional statement takes one of two possible actions
		based upon the result of a test. For example 
		\lstinline?if(a<b) ... else ... ? is a typical conditional 
		statement},
	plural=conditional statements,
	sort=conditional statement
}

\newglossaryentry{boolean expressions}
{
	name=boolean expressions,
	description={A boolean expression is an expression that has only two
		possible values: \lstinline?true? or \lstinline?false?.
		They are often controlling conditional statements. For example
		an \lstinline?if(a<b)? can only return a \lstinline?true? or
		\lstinline?false?},
	plural=boolean expressions,
	sort=boolean expression
}

\newglossaryentry{local variable}
{
	name=local variable,
	description={A local variable is a variable declared and used within a
		single method. Its scope and lifetime are limited to that
		specific method they're defined in. A special variant of local
		variables are actual parameters},
	plural=local variables,
	sort=local variable
}

\newglossaryentry{abstraction}
{
	name=abstraction,
	description={Abstraction describes the ability to ignore details and
		focus attention on a higher level of a problem. As an example
		think about an car as a Parking-Boy. You would ignore how
		many seats the cas has, but not how big it is, because it's
		relevant for your task},
	plural=abstractions,
	sort=abstraction
}

\newglossaryentry{modularization}
{
	name=modularization,
	description={Modularization is the process of dividing a whole into
		well-defined parts that can be build and examined seperately
		and that interact in well defined ways. For example a car as
		a whole entitiy can be divided into modules such as the
		engine, seats, radio, wheels and so on},
	plural=modularization,
	sort=modularization
}

\newglossaryentry{non-primitive types}
{
	name=non-primitive types,
	description={Java has eight primitive types (
		\lstinline?boolean?, \lstinline?char?,
		\lstinline?byte?, \lstinline?short?, \lstinline?int?, 
		\lstinline?long?, \lstinline?float?, \lstinline?double?) and
		gives the programmer the ability to define own types of a
		more complex manner. For example a class defines a new type 
		with the name of the class. Variables that have a class as
		their type can store objects of that class. A popular 
		example of such a type is \lstinline?String? which in fact
		is a class},
}

\newglossaryentry{primitive types}
{
	name=primitive types,
	description={The primitive types in Java are the non-object types
		\lstinline?boolean?, \lstinline?char?,
		\lstinline?byte?, \lstinline?short?, \lstinline?int?, 
		\lstinline?long?, \lstinline?float?, \lstinline?double?.
		An important characteristic to primitive-types is, that they
		don't have methods},
	plural=primitive types,
	sort=primitive types
}

\newglossaryentry{object reference}
{
	name=object reference,
	description={Variables of an object type (non-primitive type) always
		store references to objects},
	plural=object references,
	sort=object reference
}

\newglossaryentry{overloading}
{
	name=overloading,
	description={In Java sources, classes may contain multiple constructors,
		methods and variables (variable vs. parameter) with the same 
		name. This is called overloading. In Java there is a keyword 
		\lstinline?this? to
		specify the variables so that the compiler can differ them}
	plural=overloadings,
	sort=overloading
}

\newglossaryentry{collection}
{
	name=collection,
	description={A collection can store an arbitrary number of other 
		objects. Common variants for collections in Java are the
		\lstinline?ArrayList?-Objects and arrays},
	plural=collections,
	sort=collection
}

\newglossaryentry{loop}
{
	name=loop,
	description={A loop is a functionality that is given by the 
		elementary functions of a programming language, like in Java.
		They are used to repeat a sequence of expressions (a body)
		for a number of times, coupled to one or more conditions.
		In Java there are three essential types of loops: 
		The \lstinline?while?, \lstinline?do while? and
		\lstinline?for? loop. There are also other types of loops
		like the "foreach" loop},
	plural=loops,
	sort=loop
}

\newglossaryentry{iterator}
{
	name=iterator,
	description={An iterator is an object that provides functionality to
		iterate over all elements of a collection},
	plural=iterators,
	sort=iterator
}

\newglossaryentry{array}
{
	name=array,
	description={An array is a special type of collection that can store
		a fixed number of items. These items have to be of the same 
		data type},
	plural=arrays,
	sort=array
}

\newglossaryentry{null}
{
	name=null,
	description={\lstinline?null? is a reserved word in Java (and many
		other programming languages) that indicates that a reference
		is not referencing to something, that it is showing to 
		\lstinline?null?. In Java it's used to mean "no object" 
		because a refernece variable should point to a object, if it's
		not so it's containing the reference \lstinline?null?. Also
		a filed that has not been explicitly set will contain 
		\lstinline?null? if it's not defined by an other default value
		(like \lstinline?0? for variables of type  \lstinline?int?)},
	plural=,
	sort=null
}

\newglossaryentry{library}
{
	name=library,
	description={In Java a libraray is a importable collection of classes.
		These classes are arranged in packages. A known library is 
		the so called Java standard class library, which contains
		very useful classes for almost all Java programs},
	plural=libraries,
	sort=library
}

\newglossaryentry{interface}
{
	name=iterface,
	description={An interface in Java is used to agree
		or declare signatures of methods that are shared over 
		different classes. The interface is declaring which
		methods exist or have to exist. It can be used without
		showing the implementation},
	plural=interfaces,
	sort=interface
}

\newglossaryentry{implementation}
{
	name=implementation,
	description={An implementation is a source code that is defining
		something like a class or method. Takling about methods
		you could say something like "this is the implementation
		of the method XY" pointing to a source code. The 
		implementation is a recipe how something is actually 
		done (recipe is a good analogy to source code)},
	plural=implementations,
	sort=implementation
}

\newglossaryentry{immutablity}
{
	name=immutability,
	description={In Java the expression "immutable" is mostly used to
		describe an object. It sais that the objects state or 
		contents cannot be changed once it's created. A common
		example in Java are String object which are always
		immutable. So in short this means that a immutable object
		is a unchangable or fixed object},
	plural=,
	sort=immutability
}

\newglossaryentry{random number}
{
	name={random number},
	description={Real or pure random numbers in computer programming are
		not that easy to implement, since computers operate in a
		well defined and deterministic way which makes them 
		highly predictable and that way quite the opposite of 
		everything random. A common but not equivalent alternative
		are so called pseudo-random numbers. These numbers are 
		calculated by special algorithms that try to give 
		random numbers},
	plural={random numbers},
	sort={random number},
}

\newglossaryentry{library documentation}
{
	name={library documentation},
	description={The Java class library documentation shows details about
		all classes in the library. Using this documentation is 
		essential in order to make good use of the library classes},
	plural={library documentations},
	sort={library documentation}
}

\newglossaryentry{map}
{
	name=map,
	description={A map is a collection type. It stores key-value pairs as
		entries and it stores each individual 
		element (key) at most once. It does not maintain any specific 
		order. A good use for maps is a phonebook, where every key
		(person, name) is only once in the book},
	plural=maps,
	sort=map
}

\newglossaryentry{set}
{
	name=set,
	description={A set is a collection type. It stores each individual 
		element at most once. It does not maintain any specific order.
		A good use for a set is to collect all used words in a string
		(sentence, page, book), where you want to have every word 
		only once},
	plural=sets,
	sort=set
}

\newglossaryentry{documentation}
{
	name=documentation,
	description={Documentation in programming is very important and nobody
		can achieve success in programming without a good use of
		documentation. In Java there are some conventions on 
		documentation but in general the following rule is the most
		important: Write documentation (comments) so that you 
		explain what something does but not how it does it. This is
		an important aspect of desing pattern and strategy (see
		\gls{information hiding}). 
		Java has a tool named \lstinline!javadoc!
		that is converting source comments into a HTML formatted
		documentation},
	plural=documentations,
	sort=documentation
}

\newglossaryentry{information hiding}
{
	name={information hiding},
	description={Information hiding is a principle that states that 
		internal details of an implementation should be hidden
		from the user of it. 
		This shall ensure better modularization
		and support abstraction. In Java this is done with the use
		of an \gls{access modifier} like \lstinline!private! or
		\lstinline!protected!. A synonym for information 
		hiding is \gls{encapsulation}},
	plural={},
	sort={information hiding}
}

\newglossaryentry{access modifier}
{
	name={access modifier},
	description={An access modifier defines the visibility of the 
		declared field, constructor or method. There are only four of
		them in Java: 
		\lstinline!private! (visible only from inside the same class), 
		nothing specified aka. default (visible inside the same package),
		\lstinline!protected! (visible form inside the same package 
			and subclasses) and
		\lstinline!public! (visible from everywhere)},
	plural={access modifiers},
	sort={access modifier}
}

\newglossaryentry{class variable}
{
	name={class variable},
	description={In Java a class can also have a field in contrast to 
		fields that belong to each and every object that is created.
		Such a field is called class variable or 
		\gls{static variable} and exists only once for all instances
		of that class},
	plural={class variables},
	sort={class variable}
}

\newglossaryentry{static variable}
{
	name={static variable},
	description={See \gls{class variable}},
	plural={static variables},
	sort={static variable}	
}

\newglossaryentry{waterfall}
{
	name=waterfall,
	description={Waterfall or waterfall model is a sequential design 
		process or methodology. In programming it is declaring 
		that a project is planed from the very beginning up to 
		the end in sequences},
	plural=waterfalls,
	sort=waterfall
}

\newglossaryentry{agile}
{
	name=agile,
	description={Agile or agile software development is a methodology in
		software development. In short it says that only the very next
		step in the project is planned},
	plural=,
	sort=agile
}

\newglossaryentry{error}
{
	name=error,
	description={An error in software can be of different kind, in general
		there are three types: Syntax error (spelling), semantic error
		(meaning), logic error (operational). An error is often called
		a bug},
	plural=errors,
	sort=error
}

\newglossaryentry{verification}
{
	name=verification,
	description={Verification or formal verification is a act of proving 
		or disproving an software with formal methods or mathematics.
		This is in general very complicated and therefore very
		uncommon for standard applications, since it is very 
		expensive in all kind of resources.},
	plural=verifications,
	sort=verification
}

\newglossaryentry{blackbox testing}
{
	name={blackbox testing},
	description={Blackbox testing is a testmethod in software development.
		Using this method, the test is only observing the in- and 
		output of the software that is been tested. The inner
		occurances of the tested software are not observed (see
		\gls{whitebox testing} for inner obervation) },
	plural=,
	sort={blackbox testing}
}

\newglossaryentry{whitebox testing}
{
	name={whitebox testing},
	description={Whitebox testng is a testmethod in software development.
		Using this method, the test is observing the in- and output
		as well as the inner occurances of the software that is been
		tested (see \gls{blackbox testing} for onyl in- and output
		oberservation)},
	plural=,
	sort={whitebox testing}
}

\newglossaryentry{unit test}
{
	name={unit test},
	description={Unit tests are a testmethod in software development.
		Using this test only a specific software part is tested
		completely isolated form other software components of the
		same project (for example a single class is tested). Other
		testmethods are the \gls{integration test} and the 
		\gls{system test}},
	plural={unit tests},
	sort={unit test}
}

\newglossaryentry{integration test}
{
	name={integration test},
	description={The integration test is a testmethod in software 
		development. Using this testmethod multiple software
		components are tested together (for example multiple
		calsses form a package). Other testmethods are the
		\gls{unit test} and the \gls{system test}},
	plural={integration tests},
	sort={integration test}
}

\newglossaryentry{system test}
{
	name={system test},
	description={A system test is a testmethod in software development.
		Using this testmethod a complete system is tested with all
		components. Most of these tests are done by
		\gls{blackbox testing}},
	plural={system tests},
	sort={system test}
}

\newglossaryentry{code coverage}
{
	name={code coverage},
	description={Code coverage is a testmethod in software development.
		Using this method the testspecification is made such that
		every possible path in the code is taken at least once},
	plural={code coverages},
	sort={code coverage}
}

\newglossaryentry{code review}
{
	name={code review},
	description={Code review or code inspection is a systematic
		examination of source code. Reviews are done in various 
		forms such as pair programming, informal walkthroughs, 
		and formal inspections. Usually a code review is done
		by a person that hadn't developed the code. This is rising
		the chance to find overlooked mistakes in the development
		phase of the code},
	plural={code reviews},
	sort={code review}
}

\newglossaryentry{walkthrough}
{
	name=walkthrough,
	description={A walkthrough is a examination of souce code done by
		at least two personen where one is the developer of the 
		source code and leads the examination and the other is a
		person that was not involved in the development. Usually
		the code isn't examined into details by a walkthrough},
	plural=walkthroughs,
	sort=walkthrough
}

\newglossaryentry{debugger}
{
	name=debugger,
	description={A debugger is a software tool used by programmers to
		do a examination of their own code. It allows usually a 
		step by step examination with and without breakpoints and
		supports the view on variables and other data during the
		procedure. Debuggers can be used standalone or inplemneted
		in a so called IDE (Itegrated Development Environment) like
		eclipse, NetBeans or BlueJ},
	plural=debuggers,
	sort=debugger
}

\newglossaryentry{coupling}
{
	name=coupling,
	description={Coupling describes the interconnectedness
		of classes. Classes that have a lot of dependencies to
		other classes are strong coupled. Good classes have a
		weak or loose coupling, which means that they have
		no or few dependencies to other classes. Classes with
		a weak coupling for example do communicated only over
		well defined and small interfaces.},
	plural=couplings,
	sort=coupling
}

\newglossaryentry{cohesion}
{
	name=cohesion,
	description={Cohesion describes the internal mapping as a 
		logical entity (or task). If a class for example is
		called highly cohesive, this means that the internals
		are responsible for a well defined task or entity 
		(one code -- one job)},
	plural=cohesion,
	sort=cohesion
}

\newglossaryentry{code duplication}
{
	name={code duplication},
	description={If the same or logically similar code is placed
		on differents places in a source code, we say that there
		is code duplication. It is a sign of bad design 
		(especially for bad cohesion) and has
		to be avoided since there are also issues on maintenance
		and bug appearances},
	plural={code duplication},
	sort={code duplication}
}

\newglossaryentry{encapsulation}
{
	name=encapsulation,
	description={Encapsulation is used as a synonym for 
		\gls{information hiding}. Its guidline says that only
		information about what a class can do should be visible
		to the outside, not about how it does it. This leads
		to a better design and reduces the coupling},
	plural=encapsulations,
	sort=encapsulation
}

\newglossaryentry{responsibility-driven design}
{
	name={responsibility-driven design},
	description={Responsibility-driven design expresses the idea that
		each class should be responsible for handling its own
		data},
	plural={responsibility-driven designs},
	sort={responsibility-driven design}
}

\newglossaryentry{refactoring}
{
	name=refactoring,
	description={Refactoring is the activity of restructuring
		existing code to adapt it to changed functionality and
		requirements. Most important is that the design is 
		improved and not what it does},
	plural=refactoring,
	sort=refactoring
}


