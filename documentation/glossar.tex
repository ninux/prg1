\newglossaryentry{state}
{
	name=state,
	description={A object or its status is represented by his state.
		The state is represented by the values in the fields
		(instance variables)},
	sort=state
}


\newglossaryentry{class}
{
	name=class,
	description={A class describes the kind of an object. This is done
		by giving instance variables and methods. The objects 
		represents individual instatioations of the class},
	plural=classes,
	sort=class
}

\newglossaryentry{object}
{
	name=object,
	description={An object is a instance of a class},
	plural=objects,
	sort=object
}

\newglossaryentry{method}
{
	name=method,
	description={A method is a action (function) of a specific class that
		can be invoked on an object of the given class. Objects usually
		do something when a method is invoked, so a good keyword to it
		would be \textit{what}, as most methods are named by a verb.
		The methods give the objects their own particular and
		characteristic behavior},
	plural=methods,
	sort=method
}

\newglossaryentry{parameter}
{
	name=parameter,
	description={Addition information (data) given to a method or object
		is called parameter},
	plural=parameters,
	sort=parameter
}

\newglossaryentry{signature}
{
	name=signature,
	description={The signature of a method is the part that identifies 
		it to the compiler. For example the signature of 
		\lstinline{public setSpeed(int newSpeed, int newTolerance)}
		is not the whole head of the method but the name 
		\lstinline{setSpeed} and the list of parameter-types 
		\lstinline{int ..., int ...}},
	plural=signatures,
	sort=signature
}

\newglossaryentry{type}
{
	name=type,
	description={The type defines the kind of data or value (for example 
		to a parameter, return value (see data types) or a variable},
	plural=types,
	sort=type
}

\newglossaryentry{instance}
{
	name=instance,
	description={An instance is a realisation of a class to a real object,
		so instance is a synonym to object},
	plural=instances,
	sort=instance
}

\newglossaryentry{field}
{
	name=field,
	description={Fields store data for an object to use. Fields are also
		known as instance variables.},
	plural=fields,
	sort=field
}

\newglossaryentry{constructor}
{
	name=constructor,
	description={A constructor is a special method in a class which is
		responsible to initialize objects properly. In difference to
		usual methods it has no return value and is only used once},
	plural=constructors,
	sort=constructor
}

\newglossaryentry{accessor}
{
	name=accessor,
	description={A accessor or accessor method is a method that provides
		access to information about an object's state (get-methods)},
	plural=accessors,
	sort=accessor
}

\newglossaryentry{header}
{
	name=header,
	description={A header is a part of a method. It is the part that is
		not only including the signature but the whole definition.
		Example: \lstinline{public int getAge(String name)} is the
		header whereas \lstinline{getAge(String )} is the signature},
	plural=headers,
	sort=header
}
