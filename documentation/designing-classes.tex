\section{Designing classes}


\subsection{Selfstudy-Questions OOP9}

\subsubsection{Chapter 6.1 to 6.4 - Code duplication}

\subsubsection*{Exercise 1}
\textit{Solve the exercises 6.1, 6.2, 6.4 and 6.5} \\

\subsubsection*{Exercise 2}
\textit{What is coupling and how should it be?} \\
Coupling discribes iterconnectedness of classes. This means, that the 
coupling is a description for the interface of classes or the connection
between classes. A good source code has a loose or weak coupling which means
that the implementation of a class can be changed easily without effecting
the application.

\subsubsection*{Exercise 3}
\textit{What is cohesion and how should it be?} \\
Cohesion is a term that describes how well a unit (package, class, method) 
maps to a logical task or entity. Ideally, one unit of code should be 
responsible for one cohesive task (one code -- one job).

\subsubsection*{Exercise 4}
\textit{What's the problem with code duplication?} \\
The problem at code duplication is that the maintenance of an application is
going to be ineffective and buggy. Especially the danger of changing not all
of the duplicates can lead to painful bugs that are hard to find. After all
code duplication is also just a sign for bad design, so it has to be avoided.

\subsubsection*{Exercise 5}
\textit{Code duplication is a symptom for what? Bad coupling or bad cohesion? } \\
Code duplication is usually a sign of bad cohesion because one job should be done
by one code. 

\subsubsection{Chapter 6.5 to 6.11 - Cohesion}

\subsubsection*{Exercise 6}
\textit{Solve the exercises 6.6 to 6.8, 6.11, 6.14, 6.16 and 6.17} \\

\subsubsection*{Exercise 7}
\textit{The Room class is now managing neighboring rooms with a HashMap.
To do so the program had to be changed at a lot of different places. 
Is this an evidence of strong or weak coupling?} \\
This is a clear evidence of tight or strong coupling.

\subsubsection*{Exercise 8}
\textit{What is information hiding or encapsulation and what's the effect
of this fundamental principle?} \\
Information hiding or encapsulation is a fundamental priciple of good 
class design. In short it says that a user of a class should only know
that much that is needed to use it. This means a class should only
provide information about what it does and not how it does that.
The effect of this priciple is that we can change the implementation
of something without effecting the rest of the application, so it 
supports automatically a weak or loose coupling.

\subsubsection*{Exercise 9}
\textit{What's the meaning of the concept "localizing change"?} \\
The concept of "localizing change" says that it should be clear where to make
changes by improving the code. Ideally, only a single class needs to be changed
to make a modification. This is primarily achieved by following the general
design rules such as responsibility-driven design, loose coupling and strong 
cohesion.

\subsubsection*{Exercise 10}
\textit{What's the difference between explicit and implicit coupling?} \\
Explicit coupling is a coupling that is obvious, not just for a human reader
of the code but also for the compiler. Implicit coupling is the worst kind
of a strong coupling because it is not obvious, typically neither for a 
human reader nor the compiler. This is often a symptom of not following
the rule of responsibility-driven design.

\subsubsection*{Exercise 11}
\textit{Do cohesive methods make also sense?} \\
Of course they make sense. It's just an other level of code base but
the rule is the same either it's a method, class, package or application.
The cohesion should be as strong as possible in the following manner:
\begin{itemize}
	\item method -- very strong
	\item class -- less stronger than a method
	\item package -- less stronger than a class
	\item application -- less stronger than a package
\end{itemize} 

\subsubsection*{Exercise 12}
\textit{What are the two most powerful benefits of strong cohesion?} \\
The two most powerful benefits of strong cohesion are 
\begin{itemize}
	\item readability -- easy to understand and to maintain
	\item reuse -- one particular job can be used more than once 
\end{itemize}

\subsubsection{Chapter 6.12 to 6.14 - Design guidelines}

\subsubsection*{Exercise 13}
\textit{Solve the exercise 6.27} \\

\subsubsection*{Exercise 14}
\textit{What's the reason for refactoring?} \\
The reasons for refactoring are mostly changed functionality and
requirements. The refactoring itself is a process of rethinking
and redesigning existing classes and methods.

\subsubsection*{Exercise 15}
\textit{Describe the method or the steps at a refactoring.} \\
The refactoring follows usually the following steps:
\begin{enumerate}
	\item 	Change the internal structure but don't add new 
		functionality to the code.
		After the restructuration everything should work as before.
		Therefore the previous test should be ran again.
	\item	Since the restructuration is done and the tests gone well,
		the new functionalities can be implemented. Of course the
		tests should be ran again to prove the operation of the new
		developed code.
\end{enumerate}

\subsubsection*{Exercise 16}
\textit{At which point is a method too long?} \\
A method is too long if the cohesion is not strong. In such a case we 
should consider to make helper methods to outsource details from a 
method. An other unwritten rule is to check the indentations. If the code
has an indentation degree far more than three, you should probably change
your code. 

Simplified one could say "a method is too long if it does more than one 
logical task".

\subsubsection*{Exercise 17}
\textit{At which point is a class too complex?} \\
A class is to complex if there are things that don't belong there following
the responsibility-driven design rule. This leads to a creation of a new 
class or making the changes into an other class.

Simplified one could say "a class is too complex if it represents more than
one logical entity".

\subsection{Selfstudy-Questions ALG4}

\subsubsection*{Exercise 18}
\textit{Illustrate the difference between satble and instabe sorting with a 
simple example.} \\
Stable sorting means that the order of same keys is preserved in the
result and vice versa. See the following example:

\begin{table}[h!]
	\centering
	\begin{tabular}{l c c c c c}
		array	& $[r_0]$ & $[r_1]$ & $[r_2]$ & $[r_3]$ & $[r_4]$ \\ 
		\hline
		unsorted 	& $1$ & $12_1$ & $4$ & $12_2$ & $3$ \\
		stable sorted 	& $1$ & $3$ & $4$ & $12_1$ & $12_2$ \\
		instable sorted	& $1$ & $3$ & $4$ & $12_2$ & $12_1$
	\end{tabular}
	\caption{Comparison of stable and unstable sorting.}
\end{table}

\subsubsection*{Exercise 19}
\textit{What's the effort for simple or direct sorting algorithms?} \\
Simple or direct sorting algorithms have an effort of $O(n^2)$

\subsubsection*{Exercise 20}
\textit{What's the effort for higher sorting algorithms?} \\
Higher sorting algorithms have an effort of $O(n \cdot log(n))$

\subsubsection*{Exercise 21}
\textit{Enumerate three simple or direct sorting algorithms.} \\

\begin{table}[h!]
	\centering
	\begin{tabular}{l l l}
		Name		& type		& $O$ \\
		\hline
		Insertion sort	& stable	& $O(n^2)$ \\
		Selection sort	& instable	& $O(n^2)$ \\
		Shellsort	& instable	& $O(n^2)$ \\
		Bubble sort	& stable	& $O(n^2)$ 
	\end{tabular}
	\caption{Basic sorting algorithms}
	\label{table:sorting-algorithms}
\end{table}

\subsubsection*{Exercise 22}
\textit{Which of the sorting algorithms that we talked about in class 
are instable?} \\
See the table \ref{table:sorting-algorithms}.

\subsubsection*{Exercise 23}
\textit{At the analysis of srting algorithms it is important to know how to
calculate the following sum. Solve the problem for $x$.} \\
\[ 1 + 2 + 3 + \dots + n =  \sum_{i=1}^{n} i = x \]
This is a limit of sequence problem which has the form
\[ S_n = \sum_{k=1}^n a_k = a_1 + a_2 + a_3 + \dots + a_n \]
This can be either geometrical or arithmetical. In this case it's an 
arithmetical one. So it has the rule 
\[ S_n = n \cdot \left( a_1 + d \cdot \frac{n-1}{2}\right) = n \cdot \frac{a_1+a_n}{2} \]
Substituting with the given values we get the solution 
\[ S_n = n \cdot \frac{1+n}{2} \]
The following examples shall give the prove
\[ \begin{array}{l l l}
	\sum_{i=1}^{n=1} &= 1 \\
	\sum_{i=1}^{n=2} &= 1 + 2 = 3 \\
	\sum_{i=1}^{n=3} &= 1 + 2 + 3 = 6 \\
	\sum_{i=1}^{n=4} &= 1 + 2 + 3 + 4 = 10 \\
\end{array} \]
So the solution of this problem is 
\[ \sum_{i=1}^{n} i = \lim_{n \rightarrow k} n \cdot \frac{1+n}{2} = \frac{1}{2}  n (1+n) \]

\subsubsection{Optional Exercises about chapter 6.13 (p. 226-231)}

\subsubsection*{Exercise 24}
\textit{What's the difference between a class type (with the keyword class)
and a enumeration type (with the keyword enum)?} \\

\subsubsection*{Exercise 25}
\textit{Every enumeration type knows a method values(). What's the return
value of that method?} \\

