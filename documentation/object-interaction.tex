\section{Object interaction}

\subsection{Selfstudy-Questions OOP4}

\subsubsection{Chapter 3.6 - Class diagrams vs. object diagrams}

\subsubsection*{Exercise 1}
\textit{How do you declare a referencevariable?}\\

A referencevariable is a variable that points to an object.
For example \lstinline{Account myAccount = new Account();} defines
a referencevariable \lstinline{myAccount}. This variable doesn't 
contain a value but a reference to the storage-space where the
object lays (like a pointer in C).

\subsubsection*{Exercise 2}
\textit{Draw the object diagram to the BlueJ project "house" from 
chapter 1.}\\


\subsubsection*{Exercise 3}
\textit{Draw the class diagram to the BlueJ project "house" from
chapter 1.}\\


\subsubsection*{Exercise 4}
\textit{Solve the exercises 3.1 to 3.4}\\

\subsubsection{Chapter 3.8 - The ClockDisplay source code}

\subsubsection*{Exercise 1}
\textit{Solve the exercise 3.5}\\


\subsubsection*{Exercise 2}
\textit{What is the result of the following expressions?}\\

\begin{table}[h!]
	\begin{tabular}{l c l l}
	Question & & & Result \\ \hline
	\lstinline!(3>2)! & \lstinline!^! & \lstinline!(4>5)! & \lstinline!true! \\
	\lstinline!(3<2)! & \lstinline!^! & \lstinline!(4>5)! & \lstinline!fasle! \\
	\lstinline!(3<2)! & \lstinline!&&! & \lstinline!(4>5)! & \lstinline!false! \\
	\lstinline!(3>2)! & \lstinline!||! & \lstinline!(4>5)! & \lstinline!true!\\
	\lstinline?!(3>2)? &  &  & \lstinline!false!
	\end{tabular}
\end{table}

\subsubsection*{Exercise 3}
\textit{Solve the exercises 3.6 to 3.8}\\

\textbf{3.6} Nothing happens. This implemetation is not a good idea. To improve
it we could use a error-message that is returned.

\textbf{3.7} We could not set the value to zero.

\textbf{3.8} If would be true for all inputs, because their either >0 or <limit.

\subsubsection*{Exercise 4}
\textit{Solve the exercises 3.15 to 3.17 and 3.19}\\

\textbf{3.15} The modulo operator returns the remainder of an division.

\textbf{3.16} \lstinline!8%3! returns 2

\textbf{3.17} \lstinline!-10%3! returns -1, \lstinline!10%-3! returns +1.

\textbf{3.18} 5-1

\textbf{3.19} m-1

\subsubsection*{Exercise 5}
\textit{Solve the exercise 3.21}\\

\textbf{3.21} 
\begin{lstlisting}
if((value+1) < limit){
	value++;
}
else{
	value = 0;
}
\end{lstlisting}

\subsubsection{Chapter 3.9 - Objects creating objects}

\subsubsection*{Exercise 1}
\textit{Solve the exercise 3.23}\\

\textbf{3.23} The time is "00:00". The constructor is responsible for this value.

\subsubsection{Chapter 3.10 - Multiple constructors}

\subsubsection*{Exercise 1}
\textit{Create the singatures for all possible constructors which accord
with the following object-creation.}\\
\lstinline{new Student("Peter", 34);}

\begin{lstlisting}
// simple creator with no parameters
public Student()
{
	name = "No-Name";
	age = -1;
}

// creator with single-parameter name
public Student(String newName)
{
	name = newName;
	age = -1
}

// creator with single-parameter age
public Student(int newAge)
{
	name = "No-Name";
	age = newAge;
}

// creator with full paramterlist name, age
public Student(String newName, int newAge)
{
	name = newName;
	age = newAge;
}
\end{lstlisting}

\subsubsection*{Exercise 2}
\textit{Solve the exercises 3.28 and 3.29}\\

\textbf{3.28} It creates two \lstinline!NumberDisplay! objects with the 
overrolllimits 24 and 60. \\

\textbf{3.29} Because it is set by the parameters given to the constructor.

\subsubsection{Chapter 3.11 - Method calls}

\subsubsection*{Exercise 1}
\textit{Solve the exercise 3.30}\\

\textbf{3.30} 
\begin{lstlisting}
// print the Payroll-Summary on Printer p1, two-sided
p1.print("Payroll-Summary.txt", true)

// print the Phone-List on Printer p1, single-sided
p1.print("Phone-List.txt", false)

// show the status of Printer p1 on the console
System.out.println(p1.getSatus(20))

// return the status of Printer p1
p1.getStatus(10)
\end{lstlisting}

\subsubsection{Chapter 3.12 - Another example of object interaction}

\subsubsection*{Exercise 1}
\textit{Solve the exercises 3.33 and 3.34}\\

\textbf{3.33}

\lstinputlisting{../workspace/Mail-System/src/mails/Main.java}
\lstinputlisting{../workspace/Mail-System/src/mails/MailServer.java}
\lstinputlisting{../workspace/Mail-System/src/mails/MailClient.java}
\lstinputlisting{../workspace/Mail-System/src/mails/MailItem.java}


\textbf{3.34} 

\subsubsection{Chapter 3.13 - Using a debugger}

\subsubsection*{Exercise 1}
\textit{Solve the exercises 3.35 to 3.42}\\

\textbf{3.35 to 3.42} 

\lstinputlisting{../workspace/Mail-System/src/mails/Sophie.java}

\subsection{Team-Exercises}

\subsubsection{Exercise 1 - Using a debugger}

\subsubsection*{Exercise 3.43, page 90}

\subsubsection*{Exercise 3.44, page 90}

\subsubsection{Exercise 2 - Some random exercises}

\subsubsection*{Exercise 3.9, page 71}

\textit{Which of the following expressions return true?}
\begin{table}[h!]
	\begin{tabular}{l l}
		Expression & Result \\
		\hline
		\lstinline?!(4<5)? & \lstinline!true! \\
		\lstinline?!false? & \lstinline!true! \\
		\lstinline?(2>2) || ((4==4) && (1<0))? & \lstinline!false! \\
		\lstinline?(2>2) || (4==4) && (1<0)? & \lstinline!false! \\
		\lstinline?(34 != 33) && ! false? & \lstinline!true!  \\
	\end{tabular}
\end{table}

\subsubsection*{Exercise 3.10, page 71}
\textit{Write an expression using boolean variables a and b that evaluates
to true when a and b are either true or both false.}\\

\lstinline?!(a^b)?

\subsubsection*{Exercise 3.11, page 71}
\textit{Write an expression using boolean variables a and b that evaluates
to true when only one of a and b is true, and that is false if a and b are
both false or both true.}\\

\lstinline?(a^b)?

\subsubsection*{Exercise 3.12, page 71}
\textit{Consider the following expression. Write an equvalent expression
(one that evaluates true at exactly the same values for a and b) without
using the AND Operator.}\\
\lstinline?(a&&b)?\\

\subsubsection{Exercise 3 - Challanges}

\subsubsection{Exercise 4 - Programming (optional)}
\subsubsection*{Exercise 3.45, page 91}
\textit{Add a subject line for an e-mail to mail items in the mail-system
project. Make sure printing messages also prints the subject line. Modify
the mail client accodringly.}\\

\subsubsection*{Exercise 3.46, page 91}
\textit{Given the following class write some lines of java code that create
a Screen object. Then call its clear method if (and only if) its number of
pixels is greater than two million. (Don't worry about things being logical
here; the goal is only to write something that is syntactically correct - 
i.e., that would compile if we typed it in.)}\\

